%%-*-latex-*-

\begin{framed}
\begin{mathpar}
\inferrule
  {x \in \mathcal{A}\\
  \Gamma(x) \lhd (\alpha, \tau, \{(\textrm{T}, \sigma)\})\\
  \domain{\mathcal{A}} \Gamma \bouchon{15} \sigma \rightarrow
  (\overline\sigma, \domain{\mathcal{A}_0} \Gamma_0)\\
  \bouchon{0} \domain{\mathcal{A}_0 \setminus \{x\}} \Gamma_0
  \rightarrow \domain{\overline{\mathcal{A}}} \overline\Gamma}
%-----
  {\bouchon{0} \domain{\mathcal{A}} \Gamma \rightarrow
    \domain{\overline{\mathcal{A}}} \overline\Gamma \oplus x \mapsto
    (\alpha, \tau, \{(\textrm{T}, \overline\sigma)\})}

\inferrule
  {}
  {\bouchon{0} \domain{\varnothing} \Gamma \rightarrow
    \domain{\varnothing} \Gamma}
\end{mathpar}
\end{framed}

\bigskip

\begin{framed}
\begin{mathpar}
\inferrule
  {\domain{\mathcal{A}} \Gamma \bouchon{5} \nu \rightarrow
    (\overline\nu, \domain{\mathcal{A}_0} \Gamma_0)\\
  \domain{\mathcal{A}_0} \Gamma_0 \bouchon{15} \sigma \rightarrow
  (\overline\sigma, \domain{\mathcal{A}_1} \Gamma_1)}
%-----
  {\domain{\mathcal{A}} \Gamma \bouchon{15} \{\nu\} \uplus \sigma
    \rightarrow (\{\overline\nu\} \cup \overline\sigma,
    \domain{\mathcal{A}_1} \Gamma_1)}

\inferrule
  {}
  {\domain{\mathcal{A}} \Gamma \bouchon{15} \{\} \rightarrow (\{\},
    \domain{\mathcal{A}} \Gamma)}
\end{mathpar}
\end{framed}

\begin{mathparpagebreakable}
% Intersection
%
\inferrule
  {\domain{\mathcal{A}} \Gamma \bouchon{15} \sigma \rightarrow
    (\overline\sigma, \domain{\mathcal{A}_0} \Gamma_0)\\
  \domain{\mathcal{A}_0} \Gamma_0 \bouchon{5} \textsf{Inter} \, \Sigma
  \rightarrow (\textsf{Inter} \, \overline\Sigma,
  \domain{\mathcal{A}_1} \Gamma_1)}
%------
  {\domain{\mathcal{A}} \Gamma \bouchon{5} \textsf{Inter} \, (\sigma
    \Cons \Sigma) \rightarrow (\textsf{Inter} \, (\overline\sigma
    \Cons \overline\Sigma), \domain{\mathcal{A}_1} \Gamma_1)}

\inferrule
  {}
  {\domain{\mathcal{A}} \Gamma \bouchon{5} \textsf{Inter} \, \emptyL
    \rightarrow (\textsf{Inter} \, \emptyL \!, \domain{\mathcal{A}}
    \Gamma)}

% INCLUDES
%
% globalisation ET globalisation des types locaux sans leur
% �tiquetage...
%
\inferrule
  {x \not\in \mathcal{A}\\
  \overline\nu \triangleq \textsf{INCLUDES} \,\, \emptyL \,
  (\textsf{TRef} \, x) \, \{\}\\
  \domain{\mathcal{A}} \Gamma \bouchon{63} \textrm{T}' \rightarrow
  (\overline{\textrm{T}}', \domain{\overline{\mathcal{A}}}
  \overline\Gamma)\\
  \overline\gamma \triangleq x \mapsto (\emptyL\!, \tau',
  \{(\overline{\textrm{T}}', \sigma')\})}
%-----
  {\domain{\mathcal{A}} \Gamma \bouchon{5} \textsf{INCLUDES} \,\,
    \tau' \, \textrm{T}' \, \sigma' \rightarrow (\overline\nu,
    \domain{\overline{\mathcal{A}}} \overline\Gamma \oplus
    \overline\gamma)}

% SIZE
%
\inferrule
  {\domain{\mathcal{A}} \Gamma \bouchon{15} \sigma \rightarrow
    (\overline\sigma, \domain{\mathcal{A}_0} \Gamma_0)}
%-----
  {\domain{\mathcal{A}} \Gamma \bouchon{5} \textsf{SIZE} \, \sigma
    \rightarrow (\textsf{SIZE} \, \overline\sigma,
    \domain{\mathcal{A}_0} \Gamma_0)}

% FROM
%
\inferrule
  {\domain{\mathcal{A}} \Gamma \bouchon{15} \sigma \rightarrow
    (\overline\sigma, \domain{\mathcal{A}_0} \Gamma_0)}
%-----
  {\domain{\mathcal{A}} \Gamma \bouchon{5} \textsf{FROM} \, \sigma
    \rightarrow (\textsf{FROM} \, \overline\sigma,
    \domain{\mathcal{A}_0} \Gamma_0)}

% WITH COMPONENT
%
\inferrule
  {\domain{\mathcal{A}} \Gamma \bouchon{15} \sigma \rightarrow
    (\overline\sigma, \domain{\mathcal{A}_0} \Gamma_0)\\
  \overline\nu \triangleq \textsf{WITH COMPONENT} \, \overline\sigma}
%-----
  {\domain{\mathcal{A}} \Gamma \bouchon{5} \textsf{WITH COMPONENT} \,
    \sigma \rightarrow (\overline\nu, \domain{\mathcal{A}_0} \Gamma_0)}

% WITH COMPONENTS
%
\inferrule
  {\mathcal{K} \lhd \OList{$(l_j, \sigma_j, \hat\pi_j)$}{j}{p}\\
  \{\}, \domain{\mathcal{A}} \Gamma \bouchon{39}
  \SList{$\sigma_j$}{j}{p} \rightarrow
  (\SList{$\overline\sigma_j$}{j}{p}, \domain{\overline{\mathcal{A}}}
  \overline\Gamma)\\
  \overline{\mathcal{K}} \triangleq \OList{$(l_j, \overline\sigma_j,
    \hat\pi_j)$}{j}{p}\\
  \overline\nu \triangleq \textsf{WITH COMPONENTS} \, (m,
  \overline{\mathcal{K}})}
%-----
  {\domain{\mathcal{A}} \Gamma \bouchon{5} \textsf{WITH COMPONENTS} \,
    (m, \mathcal{K}) \rightarrow (\overline\nu,
    \domain{\overline{\mathcal{A}}} \overline\Gamma)}

\inferrule
  {\domain{\mathcal{A}} \Gamma \bouchon{15} \sigma \rightarrow
  (\overline\sigma, \domain{\overline{\mathcal{A}}} \overline\Gamma)\\
  \{\overline\sigma\} \cup \overline{\mathcal{S}},
  \domain{\overline{\mathcal{A}}} \overline\Gamma \bouchon{39}
  \mathcal{S} \rightarrow r}
%-----
  {\overline{\mathcal{S}}, \domain{\mathcal{A}} \Gamma \bouchon{39}
    \{\sigma\} \uplus \mathcal{S} \rightarrow r}

\inferrule
  {}
  {\overline{\mathcal{S}}, \domain{\mathcal{A}} \Gamma \bouchon{39}
    \{\} \rightarrow (\overline{\mathcal{S}}, \domain{\mathcal{A}}
    \Gamma)}
\end{mathparpagebreakable}
