%%-*-latex-*-

\vspace*{2cm}

\begin{quotation}
{\footnotesize An expedient was therefore offered, ``that since words
  are only names for things, it would be more convenient for all men
  to carry about them such things as were necessary to express a
  particular business they are to discourse on.'' [...] many of the
  most learned and wise adhere to the new scheme of expressing
  themselves by things; which has only this inconvenience attending
  it, that if a man's business be very great, and of various kinds, he
  must be obliged, in proportion, to carry a greater bundle of things
  upon his back, unless he can afford one or two strong servants to
  attend him. [...] But for short conversations, a man may carry
  implements in his pockets, and under his arms, enough to supply him;
  and in his house, he cannot be at a loss. Therefore the room where
  company meet who practise this art, is full of all things, ready at
  hand, requisite to furnish matter for this kind of artificial
  converse. Another great advantage proposed by this invention was,
  that it would serve as a universal language, to be understood in all
  civilised nations, whose goods and utensils are generally of the
  same kind, or nearly resembling, so that their uses might easily be
  comprehended.} \\ \\

\begin{flushright}
{\normalsize \textbf{Jonathan Swift}.\ \ \emph{Gulliver's Travels} \\
\emph{A Voyage to Balnibarbi}, chap.~V.}
\end{flushright}

\end{quotation}

\vspace*{2cm}

\begin{quotation}
{\footnotesize
\noindent [...] des trous, des petits trous, \\
encore des petits trous, \\
des petits trous, des petits trous, \\
toujours des petits trous... \\
Y a de quoi devenir dingue, \\
de quoi prendre un flingue...} \\ \\

\begin{flushright}
{\normalsize \textbf{Serge Gainsbourg}. \ \ \emph{Le poin�onneur des
    Lilas.}}
\end{flushright}

\end{quotation}
