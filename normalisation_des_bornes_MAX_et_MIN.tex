%%-*-latex-*-

\begin{mathparpagebreakable}
\inferrule
  {x, \textrm{T} \bouchon{61} e_0 \mid \Sigma \rightarrow
    \overline{e}_0\\
   x, \textrm{T} \bouchon{61} e_1 \mid \Sigma \rightarrow
   \overline{e}_1}
  %-----
   {x, \textrm{T} \bouchon{45} e_0 b_0 \, \textbf{..} \, b_1 e_1 \mid
     \Sigma \rightarrow \overline{e}_0 b_0 \, \textbf{..} \, b_1
     \overline{e}_1}

\inferrule
  {x, \textrm{T} \bouchon{59} \Sigma \rightarrow \textrm{min}}
  %-----
  {x, \textrm{T} \bouchon{61} \textsf{MIN} \mid \Sigma \rightarrow
    \textrm{min}}

\inferrule
  {x, \textrm{T} \bouchon{60} \Sigma \rightarrow \textrm{max}}
  %----
  {x, \textrm{T} \bouchon{61} \textsf{MAX} \mid \Sigma \rightarrow
    \textrm{max}}

\inferrule
  {e \nlhd \textsf{MIN} \mid \textsf{MAX}}
  %-----
  {x, \textrm{T} \bouchon{61} e \mid \Sigma \rightarrow e}
\end{mathparpagebreakable}

\noindent Borne inf�rieure du premier intervalle dans une liste de contraintes
\begin{mathparpagebreakable}
\inferrule
  {}
  {x, \textrm{T} \bouchon{59} \{\textsf{Value} \, (v)\} \Cons \Sigma
    \rightarrow \textsf{Val} \, (v)}

\inferrule
  {}
  {x, \textrm{T} \bouchon{59} \{e_0 b_0 \textbf{..} b_1 e_1\} \Cons
    \Sigma \rightarrow e_0}

\inferrule
  {}
  {x, \wildINTEGER \bouchon{59} \emptyL \rightarrow \textsf{MIN}}

\inferrule
  {}
  {\textrm{"REAL"}, \textrm{T} \bouchon{59} \emptyL \rightarrow
    \textsf{Val} \, (\textsf{MINUS-INFINITY})}

\inferrule
  {}
  {x, \textsf{CharString} \, (\textrm{kind}) \bouchon{59} \emptyL
    \rightarrow \textrm{lower\_bound} \, (\textrm{kind})}

\inferrule
  {\nu \nlhd \textsf{Value} \, \wild \mid \wild \wild \textbf{..}
    \wild \wild\\
   x, \textrm{T} \bouchon{59} \Sigma \rightarrow e}
  %-----
  {x, \textrm{T} \bouchon{59} \{\nu\} \Cons \Sigma \rightarrow e}
\end{mathparpagebreakable}

\noindent Borne sup�rieure du premier intervalle dans une liste de contraintes
\begin{mathparpagebreakable}
\inferrule
  {}
  {x, \textrm{T} \bouchon{60} \{\textsf{Value} \, (v)\} \Cons \Sigma
    \rightarrow \textsf{Val} \, (v)}

\inferrule
  {}
  {x, \textrm{T} \bouchon{60} \{e_0 b_0 \textbf{..} b_1 e_1\} \Cons
    \Sigma \rightarrow e_1}

\inferrule
  {}
  {x, \wildINTEGER \bouchon{60} \emptyL \rightarrow \textsf{MAX}}

\inferrule
  {}
  {\textrm{"REAL"}, \textrm{T} \bouchon{60} \emptyL \rightarrow
    \textsf{Val} \, (\textsf{PLUS-INFINITY})}

\inferrule
  {}
  {x, \textsf{CharString} \, (\textrm{kind}) \bouchon{60} \emptyL
    \rightarrow \textrm{upper\_bound} \, (\textrm{kind})}

\inferrule
  {\nu \nlhd \textsf{Value} \, \wild \mid \wild \wild \textbf{..}
    \wild \wild\\
   x, \textrm{T} \bouchon{59} \Sigma \rightarrow e}
  %-----
  {x, \textrm{T} \bouchon{59} \{\nu\} \Cons \Sigma \rightarrow e}
\end{mathparpagebreakable}
