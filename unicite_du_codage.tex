%%-*-latex-*-

\newtheorem{unicite_codage}{Th�or�me}[section]

\begin{unicite_codage}[Unicit� du codage]
\label{unicite_codage}
\ \\

\begin{tabular}{|l|}
  \hline 
  \\
  S'il existe un codage d'une valeur d'un type bien labellis�, \\
  alors ce codage est unique. \\
  \\ 
  \hline
\end{tabular}
\end{unicite_codage}

\bigskip

\noindent \textbf{Preuve du th�or�me~\ref{unicite_codage}} \\

La preuve sera la m�me que celle concernant l'unicit� du contr�le de
type (\vref{unicite_typage}) sauf que l'on utilisera respectivement
les lemmes~\vref{coding_CHOICE}, \vref{coding_SEQUENCE}, en lieu des
lemmes \vref{labels_CHOICE} et \vref{labels_SEQUENCE}.\\\
  
\noindent \textbf{Remarque} \\

Le cas du type \textsf{SET} est identique du type \textsf{SEQUENCE},
la seule difference �tant dans le r�ordonnancement �ventuel des
sous-parties du code:
\begin{mathpar}
  \inferrule
    {\domain{\mathcal{A}} \Gamma \vdash v : \tau \, (\textsf{SEQUENCE} \,
      \Phi) \Rightarrow (\psi, \textsf{Const} \, \tilde{V})\\
     \permutation{\pi}{\tilde{V}}}
    %-----
    {\domain{\mathcal{A}} \Gamma \vdash v : \tau \, (\textsf{SET} \,
      \Phi) \Rightarrow (\psi, \textsf{Const} \,  ( \pi \,\,
      (\tilde{V})))}
    \;\TirName{[21]}
\end{mathpar}

\newtheorem{coding_CHOICE}[unicite_codage]{Lemme}

\begin{coding_CHOICE}
\label{coding_CHOICE}
\begin{equation*}
\begin{array}{|ll|}
  \hline 
        & \\
  Si    & \domain{\mathcal{A}} \Gamma, \{x'\} \uplus \textrm{L} \vdashL
          \textsf{CHOICE} \, \mathcal{F} \\
  alors & \neg(\domain{\mathcal{A}} \Gamma \vdash (x' \colon v') : \tau
          \,\, (\textsf{CHOICE} \, \mathcal{F}) \Rightarrow
          \tilde{v})\\
        & \\
  \hline
 \end{array}
\end{equation*}
\end{coding_CHOICE}

\bigskip

\noindent \textbf{Preuve du lemme~\ref{coding_CHOICE}} \\

La preuve est la m�me que celle du lemme \vref{labels_CHOICE}, mais
avec l'algorithme de codage au lieu du contr�le des types. Cependant
les r�gles~\RefTirName{[13]} \vpageref{controle_des_types_13})
et~\RefTirName{[14]} \vpageref{controle_des_types_14} deviennent
respectivement~\RefTirName{[16]} \vpageref{codage_16}
et~\RefTirName{[17]}) (p.~\pageref{codage_17}).\hfill$\Box$


\newtheorem{coding_SEQUENCE}[unicite_codage]{Lemme}

\begin{coding_SEQUENCE}
\label{coding_SEQUENCE}
\begin{equation*}
\begin{array}{|ll|}
    \hline
    & \\
    Soit & v \lhd \bob [(\textsf{Some} \, x', v')] \sqcup V' \bcb \\
    Si & \domain{\mathcal{A}} \Gamma, \{x'\} \uplus \textrm{L} \vdashL \textsf{SEQUENCE} \, \Phi \\
    alors & \neg(\domain{\mathcal{A}} \Gamma \vdash  v : \textsf{SEQUENCE}
    \, \Phi \Rightarrow \tilde{v}) \\
    & \\
    \hline
\end{array}
\end{equation*}
\end{coding_SEQUENCE}

\bigskip

\noindent \textbf{Preuve du th�or�me~\ref{coding_SEQUENCE}} \\

La preuve est la m�me que celle du th�or�me~\vref{labels_SEQUENCE},
mais avec l'algorithme de codage au lieu du contr�le de
type. Cependant les r�gles~\RefTirName{[15]}
\vpageref{controle_des_types_15}, \RefTirName{[16]}
\vpageref{controle_des_types_16} et~\RefTirName{[17]}
\vpageref{controle_des_types_17} deviennent
respectivement~\RefTirName{[18]} \vpageref{codage_18},
\RefTirName{[19]} \vpageref{codage_19} et~\RefTirName{[20]}
\vpageref{codage_20}.\hfill$\Box$
