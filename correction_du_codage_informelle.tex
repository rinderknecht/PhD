%%-*-latex-*-

Nous avons d�fini jusqu'� pr�sent trois algorithmes:
\begin{itemize}

  \item Le contr�le (syntaxique) des types, au
    chapitre~\vref{definition_du_controle_des_types_informel};

  \item Le codage de r�f�rence, � la section~\vref{codage_informel});

  \item Le contr�le s�mantique des types, � la
    section~\vref{controle_semantique_des_types_informel}.

\end{itemize}
Comme nous l'avons esquiss� � la
section~\vref{definition_semantique_informelle}, nous avons d�montr�
formellement au chapitre~\vref{correction_du_codage} le th�or�me dit
de correction (s�mantique) du codage:
\begin{quote}
\em Soit une valeur~$v$ et un type~\emph{T} du noyau d'ASN.1. Soit
$\semb{$v$}{T}$ le code de~$v$ en supposant que~$v$ est de
type~\emph{T}. Si~$v$ est de type~\emph{T}, alors $\semb{$v$}{T}$ est
de type~\emph{T}.
\end{quote}
Lorsque nous disons que �~$v$~est de type~T~�, il s'agit implicitement
du succ�s du contr�le (syntaxique) des types, et lorsque nous disons
�~$\semb{$v$}{T}$~� est de type~T, il s'agit implicitement du succ�s
du contr�le s�mantique des types.

On peut comprendre ce th�or�me comme la conservation du type des
valeurs du noyau ASN.1 par le codage de r�f�rence. Autrement dit, si
nous codons une valeur~$v$ de type~T, et si nous tentons de la d�coder
en la supposant de type T, alors nous r�ussirons (n'oublions pas que,
par construction, le succ�s du contr�le s�mantique des types implique
celui du d�codage).
