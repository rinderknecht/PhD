%%-*-latex-*-

\newtheorem{Etiquettes}{D�finition}[subsection]

\begin{Etiquettes}
\ \\

\begin{tabular}{|l|}
  \hline 
  \\
  Soient deux �tiquetages $e_0$~et~$e_1$. \\
  Ils sont dits \emph{non disjoints} dans un environnement de types
  $\domain{\mathcal{A}} \Gamma$ si \\ 
  $\exists x : \mathcal{E} \times \mathcal{T}.(\domain{\mathcal{A}}
  \Gamma \vdash x \Wide e_0 \AND \domain{\mathcal{A}} \Gamma \vdash x
  \Wide e_1)$ \\
  On notera: $\domain{\mathcal{A}} \Gamma \vdash e_0 \PAR e_1$ \\
  \\
  \hline
\end{tabular}
\end{Etiquettes}

\bigskip \bigskip

\newtheorem{Reflexivite_PAR}[Etiquettes]{Propri�t�}
\begin{Reflexivite_PAR}[R�flexivit�]
\label{Reflexivite_PAR}
\begin{equation*}
\begin{array}{|l|}
  \hline
  \\
  \forall a : \mathcal{E} \times \mathcal{T}.\domain{\mathcal{A}}
  \Gamma \vdash a \PAR a \\
  \\
  \hline
 \end{array}
\end{equation*}
\end{Reflexivite_PAR}

\bigskip

\noindent \textbf{Preuve de la propri�t�~\ref{Reflexivite_PAR}} \\

\noindent D'apr�s~\ref{Reflexivite_inclusion}: $\forall a : \mathcal{E}
\times \mathcal{T}.\domain{\mathcal{A}} \Gamma \vdash a \Wide a$. D'o�:
\begin{gather*}
\forall a : \mathcal{E} \times \mathcal{T}.\domain{\mathcal{A}}
   \Gamma \vdash a \Wide a \Rightarrow \exists x : \mathcal{E}
   \times \mathcal{T}.\domain{\mathcal{A}} \Gamma \vdash x \Wide a \\
\Leftrightarrow  \exists x : \mathcal{E} \times
   \mathcal{T}.(\domain{\mathcal{A}} \Gamma \vdash x \Wide a \AND
   \domain{\mathcal{A}} \Gamma \vdash x \Wide a)
\Leftrightarrow \domain{\mathcal{A}} \Gamma \vdash a \PAR a
\end{gather*}
$\Box$

\bigskip

\newtheorem{Symetrie_PAR}[Etiquettes]{Propri�t�}
\begin{Symetrie_PAR}[Sym�trie]
\label{Symetrie_PAR}
\begin{equation*}
\begin{array}{|l|}
  \hline
  \\
  \forall a, b : \mathcal{E} \times \mathcal{T}.(\domain{\mathcal{A}}
  \Gamma \vdash a \PAR b \Rightarrow \domain{\mathcal{A}} \Gamma
  \vdash b \PAR a) \\
  \\
  \hline
 \end{array}
\end{equation*}
\end{Symetrie_PAR}

\bigskip

\noindent \textbf{Preuve de la propri�t�~\ref{Symetrie_PAR}} \\

\noindent Par d�finition:
\begin{equation*}
\begin{array}{ccc}
  \domain{\mathcal{A}} \Gamma \vdash a \PAR b & \Leftrightarrow &
  \exists x : \mathcal{E} \times \mathcal{T}.(\domain{\mathcal{A}}
  \Gamma \vdash x \Wide a \AND \domain{\mathcal{A}} \Gamma \vdash x
  \Wide b)\\
    & \Leftrightarrow & \exists x : \mathcal{E}
\times \mathcal{T}.(\domain{\mathcal{A}} \Gamma \vdash x \Wide b \AND
\domain{\mathcal{A}} \Gamma \vdash x \Wide a) \\
    & \Leftrightarrow & \domain{\mathcal{A}} \Gamma \vdash b \PAR
    a.\hfill \Box
\end{array}
\end{equation*}

\newtheorem{Theoreme_inclusion}[Etiquettes]{Th�or�me}
\begin{Theoreme_inclusion} 
\label{Theoreme_inclusion}
\begin{equation*}
\begin{tabular}{|l|}
   \hline
   \\
   Soient $\domain{\mathcal{A}} \Gamma$ un environnement de types \\
   et~$a$, $b$~et~$c$ trois �tiquetages. \\
   Alors: $\domain{\mathcal{A}} \Gamma \vdash a \Wide b \AND
   \neg(\domain{\mathcal{A}} \Gamma \vdash c \PAR b) \Rightarrow
   \neg(\domain{\mathcal{A}} \Gamma \vdash a \Wide c)$ \\ 
   \\
   \hline
 \end{tabular}
\end{equation*}
\end{Theoreme_inclusion}

\bigskip

\noindent \textbf{Preuve du th�or�me~\ref{Theoreme_inclusion}} \\

\noindent Par d�finition: $\domain{\mathcal{A}} \Gamma \vdash c \PAR b
\Leftrightarrow \exists x.(\domain{\mathcal{A}} \Gamma \vdash x \Wide
c \AND \domain{\mathcal{A}} \Gamma \vdash x \Wide b)$ \\
Alors: $\neg(\domain{\mathcal{A}} \Gamma \vdash c \PAR b)  \Leftrightarrow \forall
x.(\neg(\domain{\mathcal{A}} \Gamma \vdash x \Wide c) \OR
\neg(\domain{\mathcal{A}} \Gamma \vdash x \Wide b))$ \\
Prenons $x = a$. On a: $\neg(\domain{\mathcal{A}} \Gamma \vdash a \Wide c) \OR
\neg(\domain{\mathcal{A}} \Gamma \vdash a \Wide b)$ \\
Puisque $\domain{\mathcal{A}} \Gamma \vdash a \Wide b$, il reste:
$\neg(\domain{\mathcal{A}} \Gamma \vdash a \Wide c)$. \hfill $\Box$

\bigskip

\newtheorem{Theoreme_PAR}[Etiquettes]{Th�or�me}
\begin{Theoreme_PAR} 
\label{Theoreme_PAR}
\begin{equation*}
\begin{tabular}{|ll|}
   \hline
   & \\
   Soient & $\domain{\mathcal{A}} \Gamma$ un environnement de types \\
   et     & $e$ un �tiquetage \\
   et     & E un ensemble d'�tiquetages. \\
   Si     & $\forall x, y \in \textrm{E}.(\domain{\mathcal{A}} \Gamma
             \vdash x \PAR y \Rightarrow x = y)$ \\
   et     & $\forall z \in \textrm{E}.\neg(\domain{\mathcal{A}} \Gamma
            \vdash z \PAR e)$ \\
   alors  & $\forall x, y \in \{e\} \cup \textrm{E}.(\domain{\mathcal{A}}
            \Gamma \vdash x \PAR y \Rightarrow x = y)$ \\
   & \\
   \hline
 \end{tabular}
\end{equation*}
\end{Theoreme_PAR}

\noindent \textbf{Preuve du th�or�me~\ref{Theoreme_PAR}} \\

\noindent La propri�t� de r�flexivit�~\ref{Reflexivite_PAR} et
$\forall z \in \textrm{E}.\neg(\domain{\mathcal{A}} \Gamma \vdash
z \PAR e)$ impliquent $e \not\in \textrm{E}$. Donc $\forall x, y \in e
\cup \textrm{E}.(\domain{\mathcal{A}} \Gamma \vdash x \PAR y \Rightarrow x =
y)$ est �quivalente � la conjonction des cas suivants:
\begin{equation*}
\left\{
\begin{array}{l}
  x = e \AND \forall y \in \textrm{E}. (\domain{\mathcal{A}} \Gamma
  \vdash x \PAR y \Rightarrow x = y) \\
  \forall x,y \in \textrm{E}. (\domain{\mathcal{A}} \Gamma \vdash x \PAR y
  \Rightarrow x = y) \\
  y = e \AND \forall x \in \textrm{E}. (\domain{\mathcal{A}} \Gamma
  \vdash x \PAR y \Rightarrow x = y) \\
  x = e \AND y = e \AND (\domain{\mathcal{A}} \Gamma \vdash x \PAR y
  \Rightarrow x = y) 
\end{array}
\right.
\end{equation*}
Le second cas est la premi�re des deux hypoth�ses. Le troisi�me se
ram�ne au premier en appliquant la propri�t� de
sym�trie~\ref{Symetrie_PAR} et en renommant $x$~en~$y$. Le dernier cas
est trivial. Le premier cas est vrai � cause de la seconde hypoth�se
$\forall z \in \textrm{E}.\neg(\domain{\mathcal{A}} \Gamma \vdash
z \PAR
e)$ qui implique, avec la propri�t� de sym�trie:
$\neg(\domain{\mathcal{A}} \Gamma \vdash e \PAR y)$. $\Box$
