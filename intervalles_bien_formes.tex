%%-*-latex-*-

\begin{mathparpagebreakable}
% INTEGER
%
\inferrule
  {}
  {\bouchon{42} \textsf{MIN} \,\wild\, \textsf{\textbf{..}} \,\wild\,
    \textsf{MAX}}

\inferrule
  {}
  {\bouchon{42} \textsf{MIN} \,\wild\, \textsf{\textbf{..}} \,\wild\,
      (\textsf{Val} \, (\textsf{Int} \,\wild\!))}

\inferrule
  {}
  {\bouchon{42} (\textsf{Val} \, (\textsf{Int} \,\wild\!)) \,\wild\,
    \textsf{\textbf{..}} \,\wild\, \textsf{MAX}}

\inferrule
  {n \leqslant m}
  %----
  {\bouchon{42} (\textsf{Val} \, (\textsf{Int} \, n)) \leqslant
    \textsf{\textbf{..}} \leqslant (\textsf{Val} \, (\textsf{Int} \,
    m))}

\inferrule
  {n < m}
  %-----
  {\bouchon{42} (\textsf{Val} \, (\textsf{Int} \, n)) \leqslant
    \textsf{\textbf{..}} < (\textsf{Val} \, (\textsf{Int} \, m))}

\inferrule
  {n < m}
  %------
  {\bouchon{42} (\textsf{Val} \, (\textsf{Int} \, n)) <
    \textsf{\textbf{..}} \leqslant (\textsf{Val} \, (\textsf{Int} \,
    m))}

\inferrule
  {n+1 < m}
  %--------
  {\bouchon{42} (\textsf{Val} \, (\textsf{Int} \, n)) <
    \textsf{\textbf{..}} < (\textsf{Val} \, (\textsf{Int} \, m))}

% REAL
%
\inferrule
  {f_0 \lhd [(\textsf{Some} \, \, \textrm{"mantissa"},
              \textsf{Int}\,m_0); 
             (\textsf{Some} \, \, \textrm{"base"},
              \textsf{Int} \, b_0);\\\\
   (\textsf{Some} \, \, \textrm{"exponent"}, \textsf{Int} \, e_0)]\\
   f_1 \lhd [(\textsf{Some} \, \, \textrm{"mantissa"}, \textsf{Int} \,
     m_1); (\textsf{Some} \, \, \textrm{"base"}, \textsf{Int} \, b_1);\\\\
     (\textsf{Some} \, \, \textrm{"exponent"}, \textsf{Int} \, e_1)]\\\\
   m_0 \cdot b_0^{e_0} \leqslant m_1 \cdot b_1^{e_1}}
  %------
  {\bouchon{42} (\textsf{Val} \bob f_0 \bcb\!) \, \leqslant
    \textsf{\textbf{..}} \leqslant \, (\textsf{Val} \bob f_1 \bcb\!)}

\inferrule
  {f_0 \lhd [(\textsf{Some} \, \, \textrm{"mantissa"}, \textsf{Int} \,
      m_0); (\textsf{Some} \, \, \textrm{"base"}, \textsf{Int} \,
      b_0);\\\\
      (\textsf{Some} \, \, \textrm{"exponent"}, \textsf{Int} \,
      e_0)]\\
   f_1 \lhd [(\textsf{Some} \, \, \textrm{"mantissa"}, \textsf{Int} \,
     m_1); (\textsf{Some} \, \, \textrm{"base"}, \textsf{Int} \,
     b_1);\\\\
     (\textsf{Some} \, \, \textrm{"exponent"}, \textsf{Int} \, e_1)]\\\\\
   m_0 \cdot b_0^{e_0} < m_1 \cdot b_1^{e_1}}
  %------
  {\bouchon{42} (\textsf{Val} \bob f_0 \bcb\!) \,\wild\,
    \textsf{\textbf{..}} \,\wild\, (\textsf{Val} \bob f_1 \bcb\!)}

\inferrule
  {f_1 \lhd [(\textsf{Some} \, \, \textrm{"mantissa"}, \textsf{Int} \,
      m_1); (\textsf{Some} \, \, \textrm{"base"}, \textsf{Int} \,
      \wild\!);\\\\
      (\textsf{Some} \, \, \textrm{"exponent"}, \textsf{Int}
      \, \wild\!)]\\\\
   0 < m_1}
  %-----
  {\bouchon{42} (\textsf{Val 0.0}) \,\wild\, \textsf{\textbf{..}}
    \,\wild\, (\textsf{Val} \bob f_1 \bcb\!)}

\inferrule
  {f_0 \lhd [(\textsf{Some} \, \, \textrm{"mantissa"}, \textsf{Int} \,
      m_0); (\textsf{Some} \, \, \textrm{"base"}, \textsf{Int} \,
      \wild\!);\\\\
      (\textsf{Some} \, \, \textrm{"exponent"}, \textsf{Int}
      \, \wild\!)]\\\\
   m_0 < 0}
  %-------
  {\bouchon{42} (\textsf{Val} \bob f_0 \bcb\!) \,\wild\,
    \textsf{\textbf{..}} \,\wild\, (\textsf{Val 0.0})}

\inferrule
  {f_1 \lhd [(\textsf{Some} \, \, \textrm{"mantissa"}, \textsf{Int} \,
      \wild\!); (\textsf{Some} \, \, \textrm{"base"}, \textsf{Int} \,
      \wild\!);\\\\
      (\textsf{Some} \, \, \textrm{"exponent"}, \textsf{Int}
      \, \wild\!)]}
  %-----
  {\bouchon{42} (\textsf{Val MINUS-INFINITY}) \,\wild\,
    \textsf{\textbf{..}} \,\wild\, (\textsf{Val} \bob f_1 \bcb\!)}

\inferrule
  {f_0 \lhd [(\textsf{Some} \, \, \textrm{"mantissa"}, \textsf{Int} \,
      \wild\!); (\textsf{Some} \, \, \textrm{"base"}, \textsf{Int} \,
      \wild\!);\\\\
      (\textsf{Some} \, \, \textrm{"exponent"}, \textsf{Int}
      \, \wild\!)]}
  %------
  {\bouchon{42} (\textsf{Val} \bob f_0 \bcb\!) \,\wild\,
    \textsf{\textbf{..}} \,\wild\, (\textsf{Val PLUS-INFINITY})}

\inferrule
  {}
  {\bouchon{42} (\textsf{Val MINUS-INFINITY}) \,\wild\,
    \textsf{\textbf{..}} \,\wild\, (\textsf{Val PLUS-INFINITY})}

\inferrule
  {\textrm{String.length} \, s_0 = 1 \qquad \textrm{String.length} \, s_1 = 1\\
   \bouchon{42} \textsf{Val} \, (\textsf{Int} \, (\textrm{Char.code}
   \, (s_0.[0]))) \leqslant \textsf{\textbf{..}} \leqslant
   \textsf{Val} \, (\textsf{Int} \, (\textrm{Char.code} \, (s_1.[0])))}
  %-----
  {\bouchon{42} \textsf{Val} \, (\textsf{CharString} \, (s_0))
    \leqslant \textsf{\textbf{..}} \leqslant \textsf{Val} \,
    (\textsf{CharString} \, (s_1))}

% Identity of lower bound and upper bound
%
\inferrule
  {}
  {\bouchon{42} e \leqslant \textsf{\textbf{..}} \leqslant e}

\end{mathparpagebreakable}
