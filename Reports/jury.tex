%%-*-latex-*-

\documentclass[a4paper,11pt,twoside]{article}

% Page geometry for College Publications
%
\usepackage[bindingoffset=0cm,a4paper,centering,textheight=200mm,textwidth=120mm,includefoot,includehead,dvips]{geometry}

% Language and fonts
%
\usepackage[british]{babel}    % British English
\usepackage[T1]{fontenc}       % Required for hyphenation and \DJ
\usepackage[utf8]{inputenc}    % UTF-8 encoding
\usepackage{hyphenat}          % \hyp{} is a breakable dash
%\usepackage{url}               % To typeset URLs
\usepackage{mdwlist}           % Tight vertical spacing of list items
\usepackage{microtype}         % Microtypographic enhancements

\hfuzz 2pt % Do not report overhang less than 2pt

% Title and author
%
\title{Report on the defense of\\ Christian Rinderknecht}
\author{ Gloess \and  Guyard \and  Lorho \and 
  Mauny \and  Dubuisson \and  Hardin \and 
  Bachatène}
\date{2 December 1998}

%--------------------------------------------------------------------
%
\begin{document}


\maketitle

\thispagestyle{empty}

Christian Rinderknecht has presented his doctoral work with much
clarity and pedagogy. He answered pertinently the questions asked by
the jury, without eluding the difficult points and thus showing great
scientific rigour.

The jury has appreciated the very high quality of the contents of the
thesis. The subject addressed is vast: it consists in modelling the
language ASN.1. This language describes data transmitted over
networks, it is widely used by the telecommunication industry and is
standardised by ISO and the ITU-T. The modelling has been approached
with the respect of the customers' needs in mind: the largest part of
ASN.1 (part~1) has been dealt with, without any restriction whatsoever
and in compliance with the rationale underlying the language. The
formal development (syntax analysis, typing, semantics) has been
carried with complete rigour, without any concession of any sort. It
constitutes a correct and solid foundation for future developments of
ASN.1. The theoretical results, as well as the tools written in the
framework of this thesis, are currently in use by several French and
foreign companies. The recognition of the quality of this work is also
proved by the participation of the candidate to the ISO
standardisation committee on ASN.1

Therefore, this thesis demonstrates that formal methods can contribute
greatly to the software industry and the standardisation efforts. Its
originality and industrial impact, it has been qualified by the jury
as exemplary.

The jury brings also to notice the courage and perseverance
demonstrated by the candidate.

For all these reasons, the jury has decided to grant the honour
\emph{summa cum laude} and testifies of the great habilities of
Christian Rinderknecht for obtaining a professorship.

\end{document}
