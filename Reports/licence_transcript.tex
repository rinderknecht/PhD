%%-*-latex-*-

\documentclass[a4paper,11pt,twoside]{article}

%\usepackage[landscape]{geometry}

% Language and fonts
%
\usepackage[british]{babel}    % British English
\usepackage[T1]{fontenc}       % Required for hyphenation and \DJ
\usepackage[utf8]{inputenc}    % UTF-8 encoding
%\usepackage{hyphenat}          % \hyp{} is a breakable dash
%\usepackage{url}               % To typeset URLs
%\usepackage{mdwlist}           % Tight vertical spacing of list items
%\usepackage{microtype}         % Microtypographic enhancements
\usepackage{textcomp}          % For Numero abbreviation

\hfuzz 2pt % Do not report overhang less than 2pt

%--------------------------------------------------------------------
%
\begin{document}

\thispagestyle{empty}

\begin{flushleft}
University Pierre and Marie Curie~-~Paris~6\\
\textsc{Faculty of Informatics~22}\\
\ \\
Institut de Programmation\\
All\'ee 55-65\\
4, Place Jussieu\\
75252 Paris Cedex~05\\
\ \\
Tel: (1)~44~27~47~64\\
Fax: (1)~44~27~62~86
\end{flushleft}

\bigskip\bigskip\bigskip

\begin{center}
\textsc{\Large BACHELOR OF SCIENCE (THIRD YEAR)}\\
\ \\
\underline{\Large Transcripts}
\end{center}

\bigskip\bigskip\bigskip

I, the head of the third grade of the Bachelor of Science programme
with major in Informatics at Paris~VI, certifies hereby that
Mr~Christian Paul \textsc{Rinderknecht}, student ID
n.\kern-0.5bp\textsuperscript{o}87~00391, obtained the following
scores.

\bigskip

\begin{center}
\begin{tabular}{@{}|c|l|c|@{}}
\hline
Year & Course & Score (0--20)\\
\hline
1991 & 21201 Algorithms and Programming~1 & 10.66\\
\hline
1991 & 21202 Compilation & 12.00\\
\hline
1991 & 21203 Hardware & 14.83\\
\hline
1991 & 21204 Mathematics and Informatics & 16.00\\
\hline
1991 & 21205 Algorithms and Programming~2 & 14.33\\
\hline
1991 & 21206 Software Engineering & 16.67\\
\hline
1991 & 212N2 Software Architecture ({\small optional}) & 14.00\\
\hline
1991 & 212N7 Teleinformatics ({\small optional}) & 12.33\\
\hline
\end{tabular}
\end{center}

\bigskip

The diploma of third grade of the Bachelor in Informatics is granted
to the students who pass 6~mandatory courses and 2~optional
courses. The final score is the average of the eight courses.

\bigskip
Final score: 14.00 (with honours 70\%-80\%)
\bigskip

Paris, 11 October 1991.

\bigskip

The head\\

\medskip

[Signed by] Mr~Attal

\end{document}
