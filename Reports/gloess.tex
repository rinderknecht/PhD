%%-*-latex-*-

\documentclass[a4paper,11pt,twoside]{article}

% Page geometry for College Publications
%
\usepackage[bindingoffset=0cm,a4paper,centering,textheight=200mm,textwidth=120mm,includefoot,includehead,dvips]{geometry}

% Language and fonts
%
\usepackage[british]{babel}    % British English
\usepackage[T1]{fontenc}       % Required for hyphenation and \DJ
\usepackage[utf8]{inputenc}    % UTF-8 encoding
\usepackage{hyphenat}          % \hyp{} is a breakable dash
%\usepackage{url}               % To typeset URLs
\usepackage{mdwlist}           % Tight vertical spacing of list items
\usepackage{microtype}         % Microtypographic enhancements

\hfuzz 2pt % Do not report overhang less than 2pt

% Title and author
%
\title{\Huge Report on the thesis of\\ Christian Rinderknecht}
\author{\Large Paul Yves Gloess}
\date{9 November 1998}

%--------------------------------------------------------------------
%
\begin{document}

\maketitle

Mr~Christian Rinderknecht develops a thesis in a dissertation
consisting of 375~pages entitled \emph{Une formalisation d'ASN.1
  (Application d'une m\'ethode formelle \`a un langage de
  sp\'ecification t\'el\'ecom)}. This work is presented in two parts:
a main volume of 84~pages, with the table of contents, and an annex of
290~pages.

This work was carried out in the context of Project Cristal at INRIA:
it deals with the language ASN.1, \emph{Abstract Syntax Notation One},
in use in telecommunications to define the format of the data
exchanged over a network, independently of the language utilised by
the applications which produce or consume these data. ASN.1 is
standardised by ISO and ITU-T. The standard X.680 is presented
informally, in natural language: ASN.1 compilers, therefore the
applications which depend on them, suffer from the lack of rigour of
the standard, which is a serious practical issue because two
applications using different compilers may not rely on the same
semantics. Because the outline of the language is not clearly defined,
the compilers cannot deliver clear diagnostics to the users.

The thesis defended is that a formal approach, based on the structured
operational semantics (natural semantics) and mathematical proofs,
enables the complete clarification of X.680, and the derivation of a
specification analyser for ASN.1 which conforms to the formal
semantics. The goal was ambitious, given the complexity of the
language and its industrial ecosystem: a computer scientist would have
been tempted to reduce this complexity, which stems sometimes from
design errors, and modify the language; but this would have been
unacceptable for the industrial users of existing applications.

The decomposition of the dissertation in two volumes, chosen by the
author, is justified by the ambivalence of the audiences it addresses:
on the one hand, the telecommunication engineers and their own
culture; on the other hand, the computer scientists who study
languages, interested in the theoretical aspects of syntax and
semantics, also in the soundness and completeness of the compilation
algorithms. The first volume introduces the problems, the solutions
and the results in an informal manner, by means of examples, while
constantly referring exactly to their formal counterparts developed in
the annex.

A first general remark is that the transition between the first volume
and the annex is rather abrupt: the first volume is pleasant and
provides insights into the approach undertaken; the annex is frankly
forbidding, with hundreds of inference rules, often lain out without
comments --~210~rules only for chapter~6 (normalisation of types,
values and tags). But how could it be any different, given the extent
of the language? As a reader (suffering from an onset of
farsightedness), I feel uncomfortable to not being able to check
everything by hand, and to content myself with incursions to comfort
the intuition acquired by reading the first volume.

The approach expanded in the first volume is as follows.

Chapter~1 introduces first the essential aspects of ASN.1: the
recursivity of ASN.1 types and values; the mutual dependency between
these two concepts and the concept of constraint (which enables the
definition of subtypes). Some restrictions on the use of recursivity
are proposed, giving meaning to specifications; recursive values are
invalid. Then, by a series of normalisation steps, it is shown how to
limit the scope to a ``core ASN.1'', as expressive as ASN.1 in its
entirety.

Chapter~2 is concerned with type-checking values, leaving aside the
subtyping constraints. The reason for ignoring these constraints in a
first phase is justified by the complexity which would ensue
otherwise. A theorem (7.3.1) establishes the uniqueness of the proof
that a value is correctly typed, assuming that the types in the
specification are ``well tagged'', that is, the names of the fields
are unique in the constructs \textsc{set}, \textsc{sequence} and
\textsc{choice}.

Chapter~3 deals with the encoding of values, which involves the
implicit and explicit tags enabling to keep the information of the
field names. The theorem of ``encoding correctness'' (8.4.1) states
that the encoding preserves the type. Another theorem (8.7.1)
establishes the determinism of the ``semantic type-checking'', to wit,
the uniqueness of the proof that a code is well typed, assuming that
the types are ``well tagged'' --~this last notion is about the
implicit or implicit tags found in the types \textsc{set},
\textsc{sequence} and \textsc{choice}. Although this does not seem to
be fully proved (because that would obscure the presentation of the
semantic type-checking), it seems reasonable to deduce, under the
assumption that the encodings are unique (which is the case of the
Canonical Encoding Rules), the reversibility of the encoding: the same
value can be recovered from its encoding.

Chapter~4 completes the construction started in chapter~2, by taking
into account the subtyping constraints. A series of phases aim at
moving the constraints so they apply to the fields of the structured
types, which is made possible by extending the strict core of ASN.1
with a union operator, which acts in the end on primitive types with
simple constraints, \emph{e.g.}, \texttt{INTEGER (1|2)}. This process
supposes that the values of the specification are well typed, in the
sense of chapter~2.

It seems to us that these four chapters put forth a pertinent
approach, yielding theoretical results of practical interest, and
whose difficulty resides more in the magnitude of the language rather
than in the depth of the proofs, although the choices made by the
author for splitting the task into different stages may not have been
easy. In our opinion, the presentation would gain in clarity if the
properties expected at the start of each phase (and at the end) would
be more explicitly stated.

Here follow two observations of a more fundamental nature.

The author justly remarks (section~3.9) that the theoretical results
which he proves apply a priori to core ASN.1, not ASN.1 as a whole,
and that they extend (informally) to the complete language. In order
to formalise the proof, it should be necessary to show that all the
rewrites of Chapter~1 ``keep invariant the codes''; but this is a
drivel, because the encoding function is only defined on the core of
ASN.1.

Would it not have been useful and relatively easier to explicitly
define a set\hyp{}based semantics, with least fixed\hyp{}point, for
all ASN.1, since all the type and subtype constructors are monotone
(except the \textsc{except} recently introduced, but not taken into
account in the dissertation), and whose encoding would be like a
partial operational implementation? In that case, a definition and an
actual proof of the semantic compatibility of Chapter~1 would have
been possible. Beside, Chapter~1 does refer to such a semantics, when
discussing the issues brought by the recursive types, which seems in
tune with the intentions of the authors of ASN.1.

The formalisation of ASN.1, together with the results and proofs
mentioned above, is in itself an important contribution, even if
handwritten, thus susceptible to contain errors. Was it (or is it)
nevertheless possible to use a tool like Coq or PVS for the
formalisation and the proofs?

These few observations, or questions, do not diminish the merits of
the author, which consists in having completely designed with great
rigour an formal operational model of an industrial language, while
respecting in the smallest details the intentions of the
designers. One of the proposals of the author, the invalidation of
recursive values, is on track to being adopted by ISO. The author has
implemented his model into the analyser called \emph{Asno} (It is a
pity that it is only mentioned in the conclusion), using OCaml (15,000
lines of source code), and licensing it as free software and open
source over the internet. This efficient tool provides very clear
diagnostics, and its qualities are confirmed by industrial users, like
CNET [now Orange Labs] in France and South\hyp{}Western Bell Tech. in
the USA.

As a conclusion, I give a very favourable notice to authorise the
defence by Mr~Christian Rinderknecht of his work, in view to obtain
the title of doctor in informatics.

\bigskip\bigskip

Paul Y Gloess, Professor.

\end{document}
