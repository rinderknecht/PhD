%%-*-latex-*-

\documentclass[a4paper,11pt,twoside]{article}

% Page geometry for College Publications
%
\usepackage[bindingoffset=0cm,a4paper,centering,textheight=200mm,textwidth=120mm,includefoot,includehead,dvips]{geometry}

% Language and fonts
%
\usepackage[british]{babel}    % British English
\usepackage[T1]{fontenc}       % Required for hyphenation and \DJ
\usepackage[utf8]{inputenc}    % UTF-8 encoding
\usepackage{hyphenat}          % \hyp{} is a breakable dash
%\usepackage{url}               % To typeset URLs
\usepackage{mdwlist}           % Tight vertical spacing of list items
\usepackage{microtype}         % Microtypographic enhancements

\hfuzz 2pt % Do not report overhang less than 2pt

% Title and author
%
\title{\Huge Report on the thesis of\\ Christian Rinderknecht}
\author{\Large Jacques Guyard}
\date{8 November 1998}

%--------------------------------------------------------------------
%
\begin{document}

\maketitle

The thesis of Mr~Christian Rinderknecht belongs to the framework of
heterogeneous, distributed computer systems and their communication
protocols. It is an important domain, of increasing relevance. The
interest and originality of the work is to be situated at the
confluence of the pragmatism ruling the world of telecommunications
and the formalism of the language theorists. A language, ASN.1, and an
encoding --~both shared and standardised~-- have been defined with the
goal of enabling heterogeneous systems to communicate. Given the
complexity and the size of these standards, the existing compilers are
not conformant and many issues do show up in practice. The aim of this
work is to provide a formal definition of the notation ASN.1 with a
twofold purpose: to fix and complete the standard, on the one hand,
and to derive a software tool which is correct and trustworthy, on the
other hand.

The manuscript contains an informal presentation of the research in
80~pages, and an annex of about 280~page long, which put forward the
complete formalisation.

The introduction expresses very clearly the objectives and stakes of
the thesis and gives a very didactic summary of the completed work.

Chapter~1 delves quickly into the heart of the matter and brings forth
the first analytical phase, with transformations applicable to an
ASN.1 module in order to obtain a normalised version of it which eases
the following stages. These transformations are presented as steps
(16) that implicitly structure the chapter. As is, the presentation is
clear, but the author could have justified the number of steps and
regroup some steps in order to obtain a more explicit structure.

Chapter~2 introduces informally the type checking. The author
pre\-sents a property which exists in the standard without
justification, that is, that the types must be well tagged in order to
obtain the uniqueness of the type checking, therefore a more efficient
implementation of the algorithm.

Chapter~3 deals with the semantic or the semantic correctness of the
encoding, based on a particular encoding rule as a reference. The
uniqueness of the encoding is obtained by means of a constraint
analogous to the previous one on the well tagged types. This chapter
ends with the verification that the rewrites to the kernel indeed
preserve the codes.

Chapter~4 deals with the checking of the subtypes, which the author
has purposefully and justifiably separated from the type checking.

The conclusion summarises the objectives, mentions the built prototype
and opens a few research tracks, in particular towards a formalisation
of a more secure and compact encoding.

The annex includes six chapters which make up the benefit of the
formal logic and the proofs of the theorems stated informally in the
previous chapters.

The document in its present state is acceptable. The separation
between formal and informal aspects is a good initiative. It is
somewhat regrettable, although understandable, that the author focused
on the central part of his work, that is to say, the formalisation,
and that he only stated the problem in the introduction and gave the
results in the conclusion. It would have been interesting and useful
to have early on a chapter presenting the issues faced, including then
an overview of ASN.1, and a chapter later on where would be detailed
the system built and the applications already in use and expected.

The work must also yield publications in international venues.

\paragraph{Conclusion} This thesis, given its objectives, its approach
and results is indeniably of a scientific and practical
interest. Given the originality of the approach, the quality of the
work invested, as well as the quantity, I give a favourable notice for
the defense by Mr~Christian Rinderknecht of a thesis to obtain the
diploma of doctor of the university Paris~VI in the specialty of
informatics.

\bigskip\bigskip

Jacques Guyard

\end{document}
