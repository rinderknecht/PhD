%%-*-latex-*-

\section{R�duction des \textsf{INCLUDES}}
\label{reduction_des_INCLUDES}
%%-*-latex-*-

\begin{mathparpagebreakable}
\inferrule
  {\domain{\mathcal{A}} \Gamma \bouchon{87} \domain{\mathcal{A}}
    \Gamma \rightarrow \domain{\overline{\mathcal{A}}}
    \overline\Gamma}
  %-----
  {\bouchon{89} \domain{\mathcal{A}} \Gamma \rightarrow
    \domain{\overline{\mathcal{A}}} \overline\Gamma}

\inferrule
  {\Gamma'(x) \lhd (\alpha, \tau, \{(\textrm{T}, \sigma)\})\\
   \domain{\mathcal{A}} \Gamma, x, \alpha \bouchon{90} \sigma
   \rightarrow \overline\sigma\\
   \domain{\mathcal{A}} \Gamma \bouchon{87} \domain{\mathcal{A}'}
   \Gamma' \rightarrow \domain{\overline{\mathcal{A}}'}
   \overline\Gamma'\\
   \overline\gamma' \triangleq x \mapsto (\alpha, \tau, \{(\textrm{T},
   \overline\sigma)\})}
  %------
  {\domain{\mathcal{A}} \Gamma \bouchon{87} \domain{\{x\} \uplus
      \mathcal{A}'} \Gamma' \rightarrow
    \domain{\overline{\mathcal{A}}'} \overline\Gamma' \oplus
    \overline\gamma'}

\inferrule
  {}
  {\domain{\mathcal{A}} \Gamma \bouchon{87} \domain{\varnothing}
    \Gamma' \rightarrow \domain{\varnothing} \Gamma'}

\inferrule
  {\domain{\mathcal{A}} \Gamma, x, \alpha \bouchon{88} \nu \rightarrow
    \overline\sigma_0\\
   \domain{\mathcal{A}} \Gamma, x, \alpha \bouchon{90} \sigma
   \rightarrow \overline\sigma}
  %-----
  {\domain{\mathcal{A}} \Gamma, x, \alpha \bouchon{90} \{\nu\} \uplus
    \sigma \rightarrow \overline\sigma_0 \cup \overline\sigma}

\inferrule
  {}
  {\domain{\mathcal{A}} \Gamma, x, \alpha \bouchon{90} \{\}
    \rightarrow \{\}}

% Intersection

% ..... non-empty
%
\inferrule
  {\domain{\mathcal{A}} \Gamma, x, \alpha \bouchon{90} \sigma
    \rightarrow \overline\sigma\\
   \domain{\mathcal{A}} \Gamma, x, \alpha \bouchon{88} \textsf{Inter}
   \, \Sigma \rightarrow \{\textsf{Inter} \, \overline\Sigma\}}
  %------
  {\domain{\mathcal{A}} \Gamma, x, \alpha \bouchon{88} \textsf{Inter}
    \, (\sigma \Cons \Sigma) \rightarrow \{\textsf{Inter} \,
    (\overline\sigma \Cons \overline\Sigma)\}}

% ..... empty
%
\inferrule
  {}
  {\domain{\mathcal{A}} \Gamma, x, \alpha \bouchon{88} \textsf{Inter}
    \, \emptyL \rightarrow \{\textsf{Inter} \, \emptyL\!\}}

% INCLUDES
%
\inferrule
  {x \neq x'\\
   x' \in \mathcal{A}\\
   \Gamma(x') \lhd (\alpha', \tau', \{(\textrm{T}', \sigma')\})\\
   \alpha \listinter \alpha' \neq \{\}\\
   \domain{\mathcal{A}} \Gamma, x', \alpha' \bouchon{90} \sigma'
   \rightarrow \overline\sigma'}
  %-----
  {\domain{\mathcal{A}} \Gamma, x, \alpha \bouchon{88}
    \textsf{INCLUDES} \, \emptyL \, (\textsf{TRef} \, x') \, \{\}
    \rightarrow \overline\sigma'}

% SIZE
%
\inferrule
  {\domain{\mathcal{A}} \Gamma, \INTEGER, \emptyL \bouchon{90} \sigma
    \rightarrow \overline\sigma}
  %------
  {\domain{\mathcal{A}} \Gamma, x, \alpha \bouchon{88} \textsf{SIZE}
    \, \sigma \rightarrow \{\textsf{SIZE} \, \overline\sigma\}}

% FROM
%
\inferrule
  {\domain{\mathcal{A}} \Gamma, x, \alpha \bouchon{90} \sigma
    \rightarrow \overline\sigma}
  %------
  {\domain{\mathcal{A}} \Gamma, x, \alpha \bouchon{88} \textsf{FROM}
    \, \sigma \rightarrow \{\textsf{FROM} \, \overline\sigma\}}

% WITH COMPONENT
%
\inferrule
  {\domain{\mathcal{A}} \Gamma, x, \alpha \bouchon{90} \sigma
    \rightarrow \overline\sigma\\
   \overline\nu \triangleq \textsf{WITH COMPONENT} \,
   \overline\sigma}
  %-----
  {\domain{\mathcal{A}} \Gamma, x, \alpha \bouchon{88} \textsf{WITH
      COMPONENT} \, \sigma \rightarrow \{\overline\nu\}}

% WITH COMPONENTS
%
\inferrule
  {\forall j \in [1..p].\domain{\mathcal{A}} \Gamma, x, \alpha
    \bouchon{90} \sigma_j \rightarrow \overline\sigma_j\\
   \overline\nu \triangleq \textsf{WITH COMPONENTS} \,\, (m,
   \OList{$(l_j, \overline\sigma_j, \hat\pi_j)$}{j}{p})}
 %-----
  {\domain{\mathcal{A}} \Gamma, x, \alpha \bouchon{88} \textsf{WITH
      COMPONENTS} \,\, (m, \OList{$(l_j, \sigma_j, \hat\pi_j)$}{j}{p})
   \rightarrow \{\overline\nu\}}

% Other constraints
%
\inferrule
  {\nu \lhd \textsf{Value} \, \wild \mid \wild \wild \textbf{..} \wild
    \wild}
  %-----
  {\domain{\mathcal{A}} \Gamma, x, \alpha \bouchon{88} \nu \rightarrow
    \{\nu\}}

\end{mathparpagebreakable}


\section{Forme disjonctive}
\label{forme_disjonctive}
%%-*-latex-*-

\noindent
\textbf{let} merge = \textbf{function} \\
\hspace*{1.2mm} $(\textsf{Inter} \, \overline\Sigma_0, \textsf{Inter} \, \overline\Sigma_1)$ $\rightarrow$ $\overline\Sigma_0 \Append \overline\Sigma_1$ \\
$\mid$ $(\textsf{Inter} \, \overline\Sigma_0, \overline\nu_1)$ $\rightarrow$ $\overline\Sigma_0 \Append [\{\overline\nu_1\}]$ \\
$\mid$ $(\overline\nu_0, \textsf{Inter} \, \overline\Sigma_1)$ $\rightarrow$ $\{\overline\nu_0\} \Cons \overline\Sigma_1$ \\
$\mid$ $(\overline\nu_0, \overline\nu_1)$ $\rightarrow$ $[\{\overline\nu_0\}; \{\overline\nu_1\}]$ \\

\noindent 
\textbf{let rec} distribute\_right = \textbf{function} \\
\hspace*{1.2mm} $(\overline\nu_0, \{\overline\nu_1\} \uplus \overline\sigma_1)$ $\rightarrow$ $\{\textsf{Inter} \, (\textrm{merge} \, (\overline\nu_0, \overline\nu_1))\} \cup (\textrm{distribute\_right} \, (\overline\nu_0, \overline\sigma_1))$ \\
$\mid$ \wild \, $\rightarrow$ \{\} \\

\noindent
\textbf{let rec} distribute\_left = \textbf{function} \\
\hspace*{1.2mm} $(\{\overline\nu_0\} \uplus \overline\sigma_0, \overline\sigma_1)$ $\rightarrow$ $(\textrm{distribute\_right} \, (\overline\nu_0, \overline\sigma_1)) \cup (\textrm{distribute\_left} \, (\overline\sigma_0, \overline\sigma_1))$ \\
$\mid$ \wild \, $\rightarrow$ \{\} \\

\noindent
\textbf{let rec} disjonctive\_intersection  = \textbf{function} \\
\hspace*{1.2mm} $\sigma \Cons \Sigma$ $\rightarrow$ \textbf{let} $\overline\sigma_0$ = disjonctive\_constraint  $\sigma$ \\
\hspace*{15mm} \textbf{in} (\textbf{match} disjonctive\_intersection  $\Sigma$ \textbf{with} \\
\hspace*{23mm} \{\} $\rightarrow$ $\overline\sigma_0$ \\
\hspace*{21mm} $\mid$ $\overline\sigma_1$ $\rightarrow$ $\textrm{distribute\_left} \, (\overline\sigma_0, \overline\sigma_1)$) \\
$\mid$ $\emptyL$ $\rightarrow$ \{\} \\

\noindent
\textbf{and} disjonctive\_constraint  = \textbf{function} \\
\hspace*{1.2mm} $\{\nu\} \uplus \sigma$ $\rightarrow$ \textbf{let} $\overline\sigma_0$ = disjonctive\_base\_and\_intersection  $\nu$ \textbf{in} \\
\hspace*{22mm} \textbf{let} $\overline\sigma_1$ = disjonctive\_constraint  $\sigma$ \\
\hspace*{19mm} \textbf{in} $\overline\sigma_0 \cup \overline\sigma_1$ \\
$\mid$ \{\} $\rightarrow$ \{\} \\

\noindent
\textbf{and} disjonctive\_base\_and\_intersection  = \textbf{function} \\
\hspace*{1.2mm} \textsf{Inter} $\Sigma$ $\rightarrow$ disjonctive\_intersection $\Sigma$ \\
$\mid$ \textsf{SIZE} $\sigma$ $\rightarrow$ \textbf{let} \SList{$\overline\nu_i$}{i}{n} = disjonctive\_constraint  $\sigma$ \\
\hspace*{17mm} \textbf{in} $\bigcup_{i=1}^{n}$ \{\textsf{SIZE} \{$\overline\nu_i$\}\} \\
$\mid$ \textsf{FROM} $\sigma$ $\rightarrow$ \{\textsf{FROM} (disjonctive\_constraint  $\sigma$)\} \\
$\mid$ \textsf{INCLUDES} $\tau$ $\textrm{T}$ $\sigma$ $\rightarrow$ \{\textsf{INCLUDES} $\tau$ $\textrm{T}$ (disjonctive\_constraint  $\sigma$)\} \\
$\mid$ \textsf{WITH COMPONENT} $\sigma$ $\rightarrow$\\
\hspace*{5mm} \{\textsf{WITH COMPONENT} (disjonctive\_constraint
$\sigma$)\}\\
$\mid$ \textsf{WITH COMPONENTS} $(m, \OList{$(l_j, \sigma_j,  \hat\pi_j)$}{j}{p})$ $\rightarrow$ \\
\{\textsf{WITH COMPONENTS} $(m, \OList{$(l_j, \textrm{disjonctive\_constraint} \, (\sigma_j),  \hat\pi_j)$}{j}{p})$\} \\
$\mid$ $\nu$ $\rightarrow$ \{$\nu$\}


\section{Distribution des contraintes disjonctives}
\label{distribution_des_contraintes_disjonctives}
%%-*-latex-*-

\begin{mathparpagebreakable}
\inferrule
  {\Gamma(x) \lhd (\alpha, \tau, \{(\textrm{T}, \{\})\})\\
  \bouchon{62} \domain{\mathcal{A}} \Gamma \rightarrow
  \domain{\overline{\mathcal{A}}} \overline\Gamma}
  %-----
  {\bouchon{62} \domain{\{x\} \uplus \mathcal{A}} \Gamma \rightarrow
    \domain{\overline{\mathcal{A}}} \overline\Gamma \oplus x \mapsto
    \Gamma(x)}

\inferrule
  {\Gamma(x) \lhd (\alpha, \tau, \{(\textrm{T}, \sigma)\})\\
   \sigma \neq \{\}\\
   \bouchon{62} \domain{\mathcal{A}} \Gamma \rightarrow
   \domain{\overline{\mathcal{A}}} \overline\Gamma\\
   \SList{$\overline\nu_i$}{i}{n} \triangleq  
   \textrm{disjonctive\_constraint} \, (\sigma)\\
   \overline\gamma \triangleq x \mapsto (\alpha, \tau,
   \SList{$(\textrm{T}, \{\overline\nu_i\})$}{i}{n})}
  %-----
  {\bouchon{62} \domain{\{x\} \uplus \mathcal{A}} \Gamma \rightarrow
    \domain{\overline{\mathcal{A}}} \overline\Gamma \oplus
    \overline\gamma}

\inferrule
  {}
  {\bouchon{62} \domain{\mathcal{A}} \Gamma \rightarrow
    \domain{\mathcal{A}} \Gamma}

\end{mathparpagebreakable}



\section{Recouplage valeurs/contraintes}
\label{recouplage_valeurs_contraintes}
%%-*-latex-*-

\begin{mathparpagebreakable}
\inferrule
  {\Gamma(x) \lhd (\alpha, \tau, \{(\textrm{T}, \{\})\})\\
   \domain{\mathcal{B}} \Delta \bouchon{43} \domain{\mathcal{A}}
   \Gamma \rightarrow \domain{\overline{\mathcal{A}}} \overline\Gamma}
  %-----
  {\domain{\mathcal{B}} \Delta \bouchon{43} \domain{\{x\} \uplus
      \mathcal{A}} \Gamma \rightarrow \domain{\overline{\mathcal{A}}}
    \overline\Gamma \oplus x \mapsto \Gamma(x)}

\inferrule
  {\Gamma(x) \lhd (\alpha, \tau, \SList{$(\textrm{T},
      \{\nu_i\})$}{i}{n})\\
   \forall i \in [1..n].\domain{\mathcal{B}} \Delta \bouchon{56} \nu_i
   \rightarrow \overline\nu_i\\
   \domain{\mathcal{B}} \Delta \bouchon{43} \domain{\mathcal{A}}
   \Gamma \rightarrow \domain{\overline{\mathcal{A}}}
   \overline\Gamma\\
   \overline\gamma \triangleq x \mapsto (\alpha, \tau,
   \SList{$(\textrm{T}, \{\overline\nu_i\})$}{i}{n})}
  %-----
  {\domain{\mathcal{B}} \Delta \bouchon{43} \domain{\{x\} \uplus
      \mathcal{A}} \Gamma \rightarrow \domain{\overline{\mathcal{A}}}
    \overline\Gamma \oplus \overline\gamma}

\inferrule
  {}
  {\domain{\mathcal{B}} \Delta \bouchon{43} \domain{\varnothing}
    \Gamma \rightarrow \domain{\varnothing} \Gamma}

% Intersection

% .....non-empty
%
\inferrule
  {\domain{\mathcal{B}} \Delta \bouchon{56} \nu' \rightarrow
    \overline\nu'\\
   \domain{\mathcal{B}} \Delta \bouchon{56} \textsf{Inter} \, \Sigma'
   \rightarrow \textsf{Inter} \, \overline\Sigma'}
  %-----
  {\domain{\mathcal{B}} \Delta \bouchon{56} \textsf{Inter} \,
    (\{\nu'\}\Cons\Sigma') \rightarrow \textsf{Inter} \,
    (\{\overline\nu'\}\Cons\overline\Sigma')}

% .....empty
%
\inferrule
  {}
  {\domain{\mathcal{B}} \Delta \bouchon{56} \textsf{Inter} \, \emptyL
    \rightarrow \textsf{Inter} \, \emptyL}

% Value constraint
%
\inferrule
  {y \in \mathcal{B}\\
   \Delta(y) \lhd (\emptyL\!, \wildTRef\!, \{\}, v)}
  %-----
  {\domain{\mathcal{B}} \Delta \bouchon{56} \textsf{Value} \,
    (\textsf{VRef} \, y) \rightarrow \textsf{Value} \, (v)}

% Intervalle (1/3)
%
\inferrule
  {y_0 \in \mathcal{B} \AND \Delta(y_0) \lhd (\emptyL\!, \wildTRef\!,
    \{\}, v_0)\\
   y_1 \in \mathcal{B} \AND \Delta(y_1) \lhd (\emptyL\!, \wildTRef\!,
   \{\}, v_1)\\
   \overline\nu \triangleq (\textsf{Val} \, v_0) \, b_0 \textbf{..} b_1
   \, (\textsf{Val} \, v_1)}
  %-----
  {\domain{\mathcal{B}} \Delta \bouchon{56} (\textsf{Val} \,
    (\textsf{VRef} \, y_0)) \, b_0 \textbf{..} b_1 \, (\textsf{Val} \,
    (\textsf{VRef} \, y_1)) \rightarrow \overline\nu}

% Intervalle (2/3)
%
\inferrule
  {y_0 \in \mathcal{B}\\
   \Delta(y_0) \lhd (\emptyL\!, \wildTRef\!, \{\}, v_0)\\
   \overline\nu \triangleq (\textsf{Val} \, v_0) \, b_0 \textbf{..}
   b_1 \, e_1}
  %-----
  {\domain{\mathcal{B}} \Delta \bouchon{56} (\textsf{Val} \,
    (\textsf{VRef} \, y_0)) \, b_0 \textbf{..} b_1 \, e_1}

% Intervalle (3/3)
%
\inferrule
  {y_1 \in \Delta\\
   \Delta(y_1) \lhd (\emptyL\!, \wildTRef\!, \{\}, v_1)\\
   \overline\nu \triangleq e_0 \, b_0 \textbf{..} b_1 \, (\textsf{Val}
   \, v_1)}
  %-----
  {\domain{\mathcal{B}} \Delta \bouchon{56} e_0 \, b_0 \textbf{..} b_1
    \, (\textsf{Val} \, (\textsf{VRef} \, y_1)) \rightarrow
    \overline\nu}

% SIZE
% 
\inferrule
  {\domain{\mathcal{B}} \Delta \bouchon{56} \nu \rightarrow
    \overline\nu}
  %-----
  {\domain{\mathcal{B}} \Delta \bouchon{56} \textsf{SIZE} \, \{\nu\}
    \rightarrow \textsf{SIZE} \, \{\overline\nu\}}

% FROM
%
\inferrule
  {\domain{\mathcal{B}} \Delta \bouchon{57} \sigma \rightarrow
    \overline\sigma}
  %-----
  {\domain{\mathcal{B}} \Delta \bouchon{56} \textsf{FROM} \, \sigma
    \rightarrow \textsf{FROM} \, \overline\sigma}

% WITH COMPONENT
%
\inferrule
  {\domain{\mathcal{B}} \Delta \bouchon{57} \sigma \rightarrow
    \overline\sigma\\
   \overline\nu \triangleq \textsf{WITH COMPONENT} \,\,
   \overline\sigma}
  %-----
  {\domain{\mathcal{B}} \Delta \bouchon{56} \textsf{WITH COMPONENT}
    \,\, \sigma \rightarrow \overline\nu}

% WITH COMPONENTS
%
\inferrule
  {\forall j \in [1..p].\domain{\mathcal{B}} \Delta \bouchon{57}
    \sigma_j \rightarrow \overline\sigma_j\\
   \overline\nu \triangleq \textsf{WITH COMPONENTS} \,\, (m,
   \OList{$(l_j, \overline\sigma_j, \hat\pi_j)$}{j}{p})}
  %-----
  {\domain{\mathcal{B}} \Delta \bouchon{56} \textsf{WITH COMPONENTS}
    \,\, (m, \OList{$(l_j, \sigma_j, \hat\pi_j)$}{j}{p}) \rightarrow
    \overline\nu}

\inferrule
  {\domain{\mathcal{B}} \Delta \bouchon{56} \nu \rightarrow
    \overline\nu\\
   \domain{\mathcal{B}} \Delta \bouchon{57} \sigma \rightarrow
   \overline\sigma}
  %-----
  {\domain{\mathcal{B}} \Delta \bouchon{57} \{\nu\} \uplus \sigma
    \rightarrow \{\overline\nu\} \cup \overline\sigma}

\inferrule
  {}
  {\domain{\mathcal{B}} \Delta \bouchon{57} \{\} \rightarrow \{\}}

\end{mathparpagebreakable}


\section{Contraintes internes}
\label{normalisation_des_contraintes_internes}
%%-*-latex-*-


\subsection{R�duction partielle des contraintes internes}
\label{reduction_partielle_des_contraintes_internes}
%%-*-latex-*-

\begin{mathparpagebreakable}
\inferrule
  {\Gamma(x) \lhd (\alpha, \tau, \SList{$(\textrm{T}_k,
      \{\nu_k\})$}{k}{q})\\
   \forall k \in [1..q].x \bouchon{58} (\textrm{T}_k, \nu_k)
   \rightarrow (\overline{\textrm{T}}_k, \overline\sigma_k)\\
   \bouchon{14} \domain{\mathcal{A}} \Gamma \rightarrow
   \domain{\overline{\mathcal{A}}} \overline\Gamma\\
   \overline\gamma \triangleq x \mapsto (\alpha, \tau,
   \SList{$(\overline{\textrm{T}}_k, \overline\sigma_k)$}{k}{q})}
   %-----
  {\bouchon{14} \domain{\{x\} \uplus \mathcal{A}} \Gamma \rightarrow
    \domain{\overline{\mathcal{A}}} \overline\Gamma \oplus
    \overline\gamma}

\inferrule
  {}
  {\bouchon{14} \domain{\varnothing} \Gamma \rightarrow
    \domain{\varnothing} \Gamma}

\inferrule
  {x, \textrm{T} \bouchon{86} \nu \rightarrow \nu_0\\
   \bouchon{13} (\textrm{T}, \nu_0) \rightarrow r}
  %-----
  {x, \bouchon{58} (\textrm{T}, \nu) \rightarrow r}

\end{mathparpagebreakable}


\subsubsection{Contraintes internes des contraintes par valeurs}
\label{contraintes_internes_des_contraintes_par_valeurs}
%%-*-latex-*-

\begin{mathparpagebreakable}
% Intersection
%
\inferrule
  {x, \textrm{T} \bouchon{86} \nu' \rightarrow \overline\nu'_0\\
   x, \textrm{T} \bouchon{86} \textsf{Inter} \, \Sigma' \rightarrow
   \overline\nu'_1\\
   \overline\nu \triangleq \textsf{Inter} \, (\textrm{merge} \,
   (\overline\nu'_0, \overline\nu'_1))}
  %-----
  {x, \textrm{T} \bouchon{86} \textsf{Inter} \, (\{\nu'\}\Cons\Sigma')
    \rightarrow \overline\nu}

% SET OF / SEQUENCE OF values

% .... non-empty
%
\inferrule
  {\textrm{T} \lhd (\textsf{SEQUENCE OF} \mid \textsf{SET OF}) \,\,
    \wild \, \wild \, \wild\\
   \overline\nu_0 \triangleq \textsf{SIZE} \, \{\textsf{Value} \,
   (\textsf{Int} \, n)\}\\
   \overline\nu_1 \triangleq \textsf{WITH COMPONENT} \,
   \SList{$\textsf{Value} \, (v_i)$}{i}{n}\\
   \overline\nu \triangleq \textsf{Inter} \, [\{\nu\};
     \{\overline\nu_0\}; \{\overline\nu_1\}]}
%-----
  {x, \textrm{T} \bouchon{86} \textsf{Value} \, \bob
    \OList{$(\textsf{None}, v_i)$}{i}{n} \bcb \AS \nu \rightarrow
    \overline\nu}

% .... empty
%
\inferrule
  {\textrm{T} \lhd (\textsf{SEQUENCE OF} \mid \textsf{SET OF}) \,\,
    \wild \, \wild \, \wild\\
   \overline\nu_0 \triangleq \textsf{SIZE} \, \{\textsf{Value} \,
   (\textsf{Int} \, 0)\}\\
   \overline\nu \triangleq \textsf{Inter} \, [\{\nu\};
     \{\overline\nu_0\}]}
  %-----
  {x, \textrm{T} \bouchon{86} \textsf{Value} \, \bob \emptyL\! \bcb
    \AS \nu \rightarrow \overline\nu}

% CHOICE value
%
\inferrule
  {\textrm{T} \lhd \wildCHOICE\\
   \mathcal{K} \triangleq [(x', \{\textsf{Value} \, (v')\},
    \textsf{Some} \, \textsf{PRESENT})]\\
   \overline\nu_0 \triangleq \textsf{WITH COMPONENTS} \, \bob
   \textsf{Full} \,\, \mathcal{K} \bcb\\
   \overline\nu \triangleq \textsf{Inter} \, [\{\nu\};
     \{\overline\nu_0\}]}
  %-----
  {x, \textrm{T} \bouchon{86} \textsf{Value} \, (x' \colon v') \AS \nu
    \rightarrow \overline\nu}

% SET values
%
\inferrule
  {x, \textsf{SEQUENCE} \, \Phi \bouchon{86} \nu \rightarrow
    \overline\nu}
  %-----
  {x, \textsf{SET} \, \Phi \bouchon{86} \nu \rightarrow \overline\nu}

% SEQUENCE values

% .... Matched field
%
\inferrule
  {x \neq \REAL\\
  \textrm{T} \lhd \textsf{SEQUENCE} \, (\varphi' \Cons \Phi')\\
  \varphi' \lhd \textsf{Field} \, (l', \emptyL\!, \textsf{TRef} \, x,
  \{\}, s')\\
  V \lhd [(\textsf{Some} \, l', v')] \sqcup V'\\
  x, \textsf{SEQUENCE} \, \Phi' \bouchon{86} \textsf{Value} \, \bob V'
  \bcb \rightarrow \overline\nu'\\
  \overline\nu' \lhd \textsf{Inter} \, [\{\overline\nu'_0\};
    \{\overline\nu'_1\}]\\
  \overline\nu'_1 \lhd \textsf{WITH COMPONENTS} \, \bob \textsf{Full}
  \,\, \mathcal{K}' \bcb\\
  \mathcal{K} \triangleq (l', \{\textsf{Value} \, (v')\},
  \textsf{Some} \, \textsf{PRESENT}) \Cons \mathcal{K}'\\
  \overline\nu_1 \triangleq \textsf{WITH COMPONENTS} \, \bob
  \textsf{Full} \,\, \mathcal{K} \bcb\\
  \overline\nu \triangleq \textsf{Inter} \, [\{\nu\};
    \{\overline\nu_1\}]}
  %-------
  {x, \textrm{T} \bouchon{86} \textsf{Value} \, \bob V \bcb \AS \nu
    \rightarrow \overline\nu}

% .... Unmatched field
%
\inferrule
  {x \neq \REAL\\
   \textrm{T} \lhd \textsf{SEQUENCE} \, (\varphi' \Cons \Phi')\
   \varphi' \lhd \textsf{Field} \, (l', \emptyL\!, \textsf{TRef} \, x,
   \{\}, s')\\
   \forall (\textsf{Some} \, x', v') \listin V.x' \neq l'\\
   x, \textsf{SEQUENCE} \, \Phi' \bouchon{86} \nu \rightarrow
   \overline\nu'\\
   \overline\nu' \lhd \textsf{Inter} \, [\{\overline\nu'_0\};
     \{\overline\nu'_1\}]\\
   \overline\nu'_1 \lhd \textsf{WITH COMPONENTS} \, \bob \textsf{Full}
   \,\, \mathcal{K}' \bcb \\
   \mathcal{K} \triangleq (l', \{\}, \textsf{Some} \, \textsf{ABSENT})
   \Cons \mathcal{K}' \\
   \overline\nu_1 \triangleq \textsf{WITH COMPONENTS} \, \bob
   \textsf{Full} \,\, \mathcal{K} \bcb \\
   \overline\nu \triangleq \textsf{Inter} \, [\{\nu\}; \{\overline\nu_1\}]}
  %-----
  {x, \textrm{T} \bouchon{86} \textsf{Value} \, \bob V \bcb \AS \nu
    \rightarrow \overline\nu}

% ... Empty SEQUENCE value of empty SEQUENCE type
%
\inferrule
  {\textrm{T} \lhd \emptySEQUENCE\\
   \overline\nu_1 \triangleq \textsf{WITH COMPONENTS} \, \bob
   \textsf{Full} \,\, \emptyL\! \bcb\\
   \overline\nu \triangleq \textsf{Inter} \, [\{\nu\}; \{\overline\nu_1\}]}
  %-----
  {x, \textrm{T} \bouchon{86} \textsf{Value} \, \bob \emptyL\! \bcb
    \AS \nu \rightarrow \overline\nu}

% REAL
%
\inferrule
  {}
  {\REAL, \textrm{T} \bouchon{86} \nu \rightarrow \nu}

% Others
%
\inferrule
  {\textrm{T} \nlhd \wildSEQUENCEOF \mid \wildSETOF \mid
    \wildCHOICE\\\\
   \textrm{T} \nlhd \wildSET \mid \wildSEQUENCE}
  %-----
  {x, \textrm{T} \bouchon{86} \nu \rightarrow \nu}

\end{mathparpagebreakable}



\subsubsection{R�duction des contraintes internes}
\label{reduction_des_contraintes_internes}
%%-*-latex-*-

\begin{mathparpagebreakable}
% Intersection
%
\inferrule
  {\bouchon{13} (\textrm{T}, \nu') \rightarrow
    (\overline{\textrm{T}}_0, \overline\sigma_0)\\
   \bouchon{13} (\overline{\textrm{T}}_0, \textsf{Inter} \, \Sigma')
   \rightarrow (\overline{\textrm{T}}, \overline\sigma_1)\\
   \overline\Sigma \triangleq \textrm{list\_of\_set} \,
   (\overline\sigma_0 \cup \overline\sigma_1)\\
   \bouchon{16} \overline\Sigma \rightarrow \overline\nu}
  %-----
  {\bouchon{13} (\textrm{T}, \textsf{Inter} \, (\{\nu'\} \Cons
    \Sigma')) \rightarrow (\overline{\textrm{T}}, \{\overline\nu\})}

\inferrule
  {}
  {\bouchon{13} (\textrm{T}, \textsf{Inter} \, \emptyL\!) \rightarrow
    (\textrm{T}, \{\})}

% WITH COMPONENT

% SET OF
%
% .........Non empty contraint list
%
\inferrule
  {\textrm{T} \lhd \textsf{SET OF} \,\, \emptyL \, (\textsf{TRef} \,
    x) \, \sigma'\\
   \overline{\textrm{T}} \triangleq \textsf{SET OF} \,\, \emptyL \,
   (\textsf{TRef} \, x) \, (\textsf{Inter} \, [\sigma'; \sigma])}
  %-----
  {\bouchon{13} (\textrm{T}, \textsf{WITH COMPONENT} \,\, \sigma)
    \rightarrow (\overline{\textrm{T}}, \{\})}

% SEQUENCE OF
%
% .........Non empty contraint list
%
\inferrule
  {\textrm{T} \lhd \textsf{SEQUENCE OF} \,\, \emptyL \, (\textsf{TRef}
    \, x) \, \sigma'\\
    \overline{\textrm{T}} \triangleq \textsf{SEQUENCE OF} \,\, \emptyL
    \, (\textsf{TRef} \, x) \, (\textsf{Inter} \, [\sigma'; \sigma])}
  %-------
  {\bouchon{13} (\textrm{T}, \textsf{WITH COMPONENT} \,\, \sigma)
    \rightarrow (\overline{\textrm{T}}, \{\})}

% WITH COMPONENTS
%
\inferrule
  {\emptyL, m \bouchon{12} (\textrm{T}, \mathcal{K}) \rightarrow
    \overline{\textrm{T}}}
  %------
  {\bouchon{13} (\textrm{T}, \textsf{WITH COMPONENTS} \, \bob m \,\,
    \mathcal{K}\bcb \! ) \rightarrow (\overline{\textrm{T}}, \{\})}

% Others
%
\inferrule
  {\nu \nlhd \textsf{Inter} \, \wild \mid \textsf{WITH COMPONENT} \,
    \wild \mid \textsf{WITH COMPONENTS} \, \wild}
  %-----
  {\bouchon{13} (\textrm{T}, \nu) \rightarrow (\textrm{T}, \nu)}

\inferrule
  {}
  {\bouchon{16} \emptyL \rightarrow \{\}}

\inferrule
  {}
  {\bouchon{16} [\overline\sigma] \rightarrow \overline\sigma}

\inferrule
  {\overline\Sigma \lhd \wild\!\Cons\wild\!\Cons\wild}
  %------
  {\bouchon{16} \overline\Sigma \rightarrow \{\textsf{Inter} \,
    \overline\Sigma\}}

\end{mathparpagebreakable}

%%-*-latex-*-

\begin{mathparpagebreakable}
% WITH COMPONENTS
%
% ........CHOICE with non empty constraint list
%
%         ....... None constraint
%
\inferrule
  {f' \lhd (l', \tau', \textrm{T}', \{\}, s')\\
   k' \lhd (l', \sigma, \textsf{None})\\
   \overline{f}' \triangleq (l', \tau', \textrm{T}', \sigma, s')\\
   \overline{\mathcal{F}} \Append [\overline{f}'], m \bouchon{12}
   \textsf{CHOICE} \, (\mathcal{F}') \,\|\, \mathcal{K}' \rightarrow
   \overline{\textrm{T}}}
  %-----
  {\overline{\mathcal{F}}, m \bouchon{12} \textsf{CHOICE} \, ([f']
    \sqcup \mathcal{F}') \,\|\, [k'] \sqcup \mathcal{K}' \rightarrow
    \overline{\textrm{T}}}

%         .........PRESENT constraint
%
\inferrule
  {(l', \tau', \textrm{T}', \{\}, s') \listin \mathcal{F}\\
   k' \lhd (l', \sigma, \textsf{Some PRESENT})\\
   (\wild\!, \wild\!, \textsf{Some PRESENT}) \not\listin
   \mathcal{K}'\\
   \overline{f}' \triangleq (l', \tau', \textrm{T}', \sigma, s')}
  %-----
  {\overline{\mathcal{F}}, m \bouchon{12} \textsf{CHOICE} \,
    (\mathcal{F}) \,\|\, [k'] \sqcup \mathcal{K}' \rightarrow
    \textsf{CHOICE} \, [\overline{f}']}

%         .........ABSENT constraint
%
\inferrule
  {f' \lhd (l', \tau', \textrm{T}', \sigma', s')\\
   k' \lhd (l', \wild\!, \textsf{Some ABSENT})\\
   \overline{\mathcal{F}} \nlhd \emptyL \OR \mathcal{F}' \nlhd
   \emptyL\\
   \overline{\mathcal{F}}, m \bouchon{12} \textsf{CHOICE} \,
   (\mathcal{F}') \,\|\, \mathcal{K}' \rightarrow
   \overline{\textrm{T}}}
  %------
  {\overline{\mathcal{F}}, m \bouchon{12} \textsf{CHOICE} \, ([f']
    \sqcup \mathcal{F}') \,\|\, [k'] \sqcup \mathcal{K}' \rightarrow
    \overline{\textrm{T}}}

%         ......... ``None'' constraint
%
% ........CHOICE with empty constraint list and partial spec
%
\inferrule
  {}
  {\overline{\mathcal{F}}, \textsf{Partial} \bouchon{12}
    \textsf{CHOICE} \, (\mathcal{F}) \,\|\, \emptyL \rightarrow
    \textsf{CHOICE} \, (\overline{\mathcal{F}} \Append \mathcal{F})}

% ........CHOICE with empty constraint list and full spec
%
\inferrule
  {}
  {\overline{\mathcal{F}}, \textsf{Full} \bouchon{12} \wildCHOICE
    \,\|\, \emptyL \rightarrow \textsf{CHOICE} \,
    \overline{\mathcal{F}}}

%......... CHOICE with non empty constraint list but no constraint for the current field
%
\inferrule
  {f' \lhd (l', \tau', \textrm{T}', \sigma', s')\\
   (l', \wild\!, \wild\!) \not\in \mathcal{K}\\
   \overline{\mathcal{F}} \Append [f'], m \bouchon{12} \textsf{CHOICE}
   \, (\mathcal{F}') \,\|\, \mathcal{K} \rightarrow
   \overline{\textrm{T}}}
  %-----
  {\overline{\mathcal{F}}, m \bouchon{12} \textsf{CHOICE} \, ([f']
    \sqcup \mathcal{F}') \,\|\, \mathcal{K} \rightarrow
    \overline{\textrm{T}}}

% ........SEQUENCE with non empty constraint list
%
%         ........and ABSENT constraint
%
\inferrule
  {\varphi' \lhd \textsf{Field} \, (l', \tau', \textrm{T}', \sigma',
    s')\\
   s' \lhd \textsf{Some OPTIONAL}\\
   k' \lhd (l', \sigma, \textsf{Some ABSENT})\\
   \overline{\mathcal{F}}, m \bouchon{12} \textsf{SEQUENCE} \, \Phi'
   \,\|\, \mathcal{K}' \rightarrow \overline{\textrm{T}}}
  %-----
  {\overline{\mathcal{F}}, m \bouchon{12} \textsf{SEQUENCE} \,
    (\varphi' \Cons \Phi') \,\|\, k' \Cons \mathcal{K}' \rightarrow
    \overline{\textrm{T}}}

%         ........ ``None'' constraint AND Partial Spec
%
\inferrule
  {\varphi'\lhd \textsf{Field} \, (l', \tau', \textrm{T}', \{\},
    s')\\
   k' \lhd (l', \sigma, \textsf{None})\\
   \overline{f}' \triangleq (l', \tau', \textrm{T}', \sigma, s')\\
   \overline{\mathcal{F}} \Append [\overline{f}'], \textsf{Partial}
   \bouchon{12} \textsf{SEQUENCE} \, (\Phi') \,\|\, \mathcal{K}'
   \rightarrow \overline{\textrm{T}}}
  %----
  {\overline{\mathcal{F}}, \textsf{Partial} \bouchon{12}
    \textsf{SEQUENCE} \, (\varphi' \Cons \Phi') \,\|\, k' \Cons
    \mathcal{K}' \rightarrow \overline{\textrm{T}}}

%         ........ ``None'' constraint AND Full Spec
%
\inferrule
  {\textrm{T} \lhd \textsf{SEQUENCE} \, (\varphi' \Cons \Phi')\\
   \varphi' \lhd \textsf{Field} \, (l', \tau', \textrm{T}', \{\}, s')\\
   s' \lhd \textsf{Some} \, \textsf{OPTIONAL}\\
   k' \lhd (l', \sigma, \textsf{None})\\
   \overline{\mathcal{K}} \triangleq (l', \sigma, \textsf{Some
     PRESENT}) \Cons \mathcal{K}'\\
   \overline{\mathcal{F}}, \textsf{Full} \bouchon{12} \textrm{T}
   \,\|\, \overline{\mathcal{K}} \rightarrow \overline{\textrm{T}}}
  %-----
  {\overline{\mathcal{F}}, \textsf{Full} \bouchon{12} \textrm{T}
    \,\|\, k' \Cons \mathcal{K}' \rightarrow \overline{\textrm{T}}}

%         ........and neither ABSENT nor ``None'' constraint
%
\inferrule
  {\varphi'\lhd \textsf{Field} \, (l', \tau', \textrm{T}', \{\},
    s')\\
   s' \lhd \textsf{Some} \, \textsf{OPTIONAL}\\
   k' \lhd (l', \sigma, \textsf{Some} \, \textsf{OPTIONAL})\\
   \overline{f}' \triangleq (l', \tau', \textrm{T}', \sigma, s')\\
   \overline{\mathcal{F}} \Append [\overline{f}'], m \bouchon{12}
   \textsf{SEQUENCE} \, (\Phi') \,\|\, \mathcal{K}' \rightarrow
   \overline{\textrm{T}}}
  %----
  {\overline{\mathcal{F}}, m \bouchon{12} \textsf{SEQUENCE} \,
    (\varphi' \Cons \Phi') \,\|\, k' \Cons \mathcal{K}' \rightarrow
    \overline{\textrm{T}}}

%         ........and neither ABSENT nor ``None'' constraint
%
\inferrule
  {\varphi'\lhd \textsf{Field} \, (l', \tau', \textrm{T}', \{\},
    s')\\
   s' \lhd \textsf{Some} \, \textsf{OPTIONAL}\\
   k' \lhd (l', \sigma, \textsf{Some} \, \textsf{PRESENT})\\
   \overline{f}' \triangleq (l', \tau', \textrm{T}', \sigma,
   \textsf{None})\\
   \overline{\mathcal{F}} \Append [\overline{f}'], m \bouchon{12}
   \textsf{SEQUENCE} \, (\Phi') \,\|\, \mathcal{K}' \rightarrow
   \overline{\textrm{T}}}
  %-----
  {\overline{\mathcal{F}}, m \bouchon{12} \textsf{SEQUENCE} \,
    (\varphi' \Cons \Phi') \,\|\, k' \Cons \mathcal{K}' \rightarrow
    \overline{\textrm{T}}}

%......... SEQUENCE with non empty constraint list but no constraint
%          for the current field 
%
\inferrule
  {\varphi' \lhd \textsf{Field} \, (f')\\
   f' \lhd (l', \tau', \textrm{T}', \sigma', s')\\
   \mathcal{K} \lhd (l, \sigma, \hat{\pi}) \Cons \mathcal{K}'\\
   l \neq l'\\
   \overline{\mathcal{F}} \Append [f'], m \bouchon{12}
   \textsf{SEQUENCE} \, (\Phi') \,\|\, \mathcal{K} \rightarrow
   \overline{\textrm{T}}}
  %-----
  {\overline{\mathcal{F}}, m \bouchon{12} \textsf{SEQUENCE} \,
    (\varphi' \Cons \Phi') \,\|\, \mathcal{K} \rightarrow
    \overline{\textrm{T}}}

% ........SEQUENCE with empty constraint list and partial spec
%
\inferrule
  {\overline\Phi \triangleq \textrm{List.map} \,\, \textsf{Field} \,\,
    \overline{\mathcal{F}}}
  %-----
  {\overline{\mathcal{F}}, \textsf{Partial} \bouchon{12}
    \textsf{SEQUENCE} \, (\Phi) \,\|\, \emptyL \rightarrow
    \textsf{SEQUENCE} \, (\overline\Phi \Append \Phi)}

% ........SEQUENCE with empty constraint list and full spec
%
% .............. non empty field list
%
% .....................OPTIONAL field
%
\inferrule
  {\varphi' \lhd \textsf{Field} \, (l', \tau', \textrm{T}', \sigma',
    s')\\
   s' \lhd \textsf{Some} \, \textsf{OPTIONAL}\\
   \overline{\mathcal{K}} \triangleq [(l', \{\}, \textsf{Some} \,
     \textsf{ABSENT})]\\
   \overline{\mathcal{F}}, \textsf{Full} \bouchon{12}
   \textsf{SEQUENCE} \, ([\varphi'] \sqcup \Phi') \,\|\,
   \overline{\mathcal{K}} \rightarrow \overline{\textrm{T}}}
  %-----
  {\overline{\mathcal{F}}, \textsf{Full} \bouchon{12}
    \textsf{SEQUENCE} \, ([\varphi'] \sqcup \Phi') \,|\, \emptyL
    \rightarrow \overline{\textrm{T}}}

% .....................non OPTIONAL field
%
\inferrule
  {\varphi' \lhd \textsf{Field} \, (f')\\
   f' \lhd (l', \tau', \textrm{T}', \sigma', s')\\
   s' \nlhd \textsf{Some} \, \textsf{OPTIONAL}\\
   \overline{\mathcal{F}} \Append [f'], \textsf{Full} \bouchon{12}
   \textsf{SEQUENCE} \, (\Phi') \,|\, \emptyL \rightarrow
   \overline{\textrm{T}}}
  %-----
  {\overline{\mathcal{F}}, \textsf{Full} \bouchon{12}
    \textsf{SEQUENCE} \, ([\varphi'] \sqcup \Phi') \,|\, \emptyL
    \rightarrow \overline{\textrm{T}}}

%............... no field
%
\inferrule
  {\overline\Phi \triangleq \textrm{List.map} \,\, \textsf{Field} \,\,
    \overline{\mathcal{F}}}
  %-----
  {\overline{\mathcal{F}}, \textsf{Full} \bouchon{12}
    \textsf{SEQUENCE} \, \emptyL \,\|\, \emptyL \rightarrow
    \textsf{SEQUENCE} \, \overline\Phi}

% SET
%
\inferrule
  {\overline{\mathcal{F}}, m \bouchon{12} \textsf{SEQUENCE} \, (\Phi)
    \,\|\, \pi \, (\mathcal{K}) \rightarrow \overline{\textrm{T}}\\
  \perm{\pi}}
  %------
  {\overline{\mathcal{F}}, m \bouchon{12} \textsf{SET} \, (\Phi)
    \,\|\, \mathcal{K} \rightarrow \overline{\textrm{T}}}
 
\end{mathparpagebreakable}




\subsection{R�duction compl�te des contraintes internes}
\label{reduction_complete_des_contraintes_internes}
%%-*-latex-*-

\begin{mathparpagebreakable}
\inferrule
  {\bouchon{19} \domain{\mathcal{A}} \Gamma \rightarrow
    \domain{\mathcal{A}} \Gamma}
  %-----
  {\bouchon{20} \domain{\mathcal{A}} \Gamma \rightarrow
    \domain{\mathcal{A}} \Gamma}

\inferrule
  {\bouchon{19} \domain{\mathcal{A}} \Gamma \rightarrow
    \domain{\mathcal{A}_0} \Gamma_0\\
   \domain{\mathcal{A}} \Gamma \neq \domain{\mathcal{A}_0} \Gamma_0\\
   \bouchon{20} \domain{\mathcal{A}_0} \Gamma_0 \rightarrow
   \domain{\mathcal{A}_1} \Gamma_1}
  %-----
  {\bouchon{20} \domain{\mathcal{A}} \Gamma \rightarrow
    \domain{\mathcal{A}_1} \Gamma_1}

\inferrule
  {\bouchon{14} \domain{\mathcal{A}} \Gamma
    \rightarrow\domain{\mathcal{A}_0}  \Gamma_0 \\
   \bouchon{65} \domain{\mathcal{A}_0} \Gamma_0 \rightarrow
   \domain{\mathcal{A}_1} \Gamma_1 \\
   \bouchon{68} \domain{\mathcal{A}_1} \Gamma_1 \rightarrow
   \domain{\mathcal{A}_2} \Gamma_2\\
   \bouchon{62} \domain{\mathcal{A}_2} \Gamma_2 \rightarrow
   \domain{\overline{\mathcal{A}}} \overline\Gamma}
  %-----
  {\bouchon{19} \domain{\mathcal{A}} \Gamma \rightarrow
    \domain{\overline{\mathcal{A}}} \overline\Gamma}

\end{mathparpagebreakable}



\section{Contraintes d'intervalle}
\label{contraintes_d_intervalle}
%%-*-latex-*-

\subsection{Normalisation des bornes finies r�elles}
\label{normalisation_des_bornes_finies_reelles}
%%-*-latex-*-

\begin{mathparpagebreakable}
% Value range
%
\inferrule
  {\bouchon{97} e_0 \rightarrow \overline{e}_0\\
   \bouchon{97} e_1 \rightarrow \overline{e}_1}
  %-----
  {\bouchon{50} e_0 \, b_0 \, \textbf{..} \, b_1 \, e_1 \rightarrow
    \overline{e}_0 \, b_0 \, \textbf{..} \, b_1 \, \overline{e}_1}

% Other
%
\inferrule
  {\nu \nlhd \wild \wild \textbf{..} \wild \wild}
  %-----
  {\bouchon{50} \nu \rightarrow \nu}

\inferrule
  {\overline{v} \triangleq \textrm{Real.normalize} \, (v)}
  %-----
  {\bouchon{97} \textsf{Val} \, (v) \rightarrow \textsf{Val} \,
    (\overline{v})}

\inferrule
  {e \nlhd \textsf{Val} \, \wild}
  %-----
  {\bouchon{97} e \rightarrow e}

\end{mathparpagebreakable}


\subsection{Normalisation des bornes \textsf{MAX} et \textsf{MIN}}
\label{normalisation_des_bornes_MAX_et_MIN}
%%-*-latex-*-

\begin{mathparpagebreakable}
\inferrule
  {x, \textrm{T} \bouchon{61} e_0 \mid \Sigma \rightarrow
    \overline{e}_0\\
   x, \textrm{T} \bouchon{61} e_1 \mid \Sigma \rightarrow
   \overline{e}_1}
  %-----
   {x, \textrm{T} \bouchon{45} e_0 b_0 \, \textbf{..} \, b_1 e_1 \mid
     \Sigma \rightarrow \overline{e}_0 b_0 \, \textbf{..} \, b_1
     \overline{e}_1}

\inferrule
  {x, \textrm{T} \bouchon{59} \Sigma \rightarrow \textrm{min}}
  %-----
  {x, \textrm{T} \bouchon{61} \textsf{MIN} \mid \Sigma \rightarrow
    \textrm{min}}

\inferrule
  {x, \textrm{T} \bouchon{60} \Sigma \rightarrow \textrm{max}}
  %----
  {x, \textrm{T} \bouchon{61} \textsf{MAX} \mid \Sigma \rightarrow
    \textrm{max}}

\inferrule
  {e \nlhd \textsf{MIN} \mid \textsf{MAX}}
  %-----
  {x, \textrm{T} \bouchon{61} e \mid \Sigma \rightarrow e}
\end{mathparpagebreakable}

\noindent Borne inf�rieure du premier intervalle dans une liste de contraintes
\begin{mathparpagebreakable}
\inferrule
  {}
  {x, \textrm{T} \bouchon{59} \{\textsf{Value} \, (v)\} \Cons \Sigma
    \rightarrow \textsf{Val} \, (v)}

\inferrule
  {}
  {x, \textrm{T} \bouchon{59} \{e_0 b_0 \textbf{..} b_1 e_1\} \Cons
    \Sigma \rightarrow e_0}

\inferrule
  {}
  {x, \wildINTEGER \bouchon{59} \emptyL \rightarrow \textsf{MIN}}

\inferrule
  {}
  {\textrm{"REAL"}, \textrm{T} \bouchon{59} \emptyL \rightarrow
    \textsf{Val} \, (\textsf{MINUS-INFINITY})}

\inferrule
  {}
  {x, \textsf{CharString} \, (\textrm{kind}) \bouchon{59} \emptyL
    \rightarrow \textrm{lower\_bound} \, (\textrm{kind})}

\inferrule
  {\nu \nlhd \textsf{Value} \, \wild \mid \wild \wild \textbf{..}
    \wild \wild\\
   x, \textrm{T} \bouchon{59} \Sigma \rightarrow e}
  %-----
  {x, \textrm{T} \bouchon{59} \{\nu\} \Cons \Sigma \rightarrow e}
\end{mathparpagebreakable}

\noindent Borne sup�rieure du premier intervalle dans une liste de contraintes
\begin{mathparpagebreakable}
\inferrule
  {}
  {x, \textrm{T} \bouchon{60} \{\textsf{Value} \, (v)\} \Cons \Sigma
    \rightarrow \textsf{Val} \, (v)}

\inferrule
  {}
  {x, \textrm{T} \bouchon{60} \{e_0 b_0 \textbf{..} b_1 e_1\} \Cons
    \Sigma \rightarrow e_1}

\inferrule
  {}
  {x, \wildINTEGER \bouchon{60} \emptyL \rightarrow \textsf{MAX}}

\inferrule
  {}
  {\textrm{"REAL"}, \textrm{T} \bouchon{60} \emptyL \rightarrow
    \textsf{Val} \, (\textsf{PLUS-INFINITY})}

\inferrule
  {}
  {x, \textsf{CharString} \, (\textrm{kind}) \bouchon{60} \emptyL
    \rightarrow \textrm{upper\_bound} \, (\textrm{kind})}

\inferrule
  {\nu \nlhd \textsf{Value} \, \wild \mid \wild \wild \textbf{..}
    \wild \wild\\
   x, \textrm{T} \bouchon{59} \Sigma \rightarrow e}
  %-----
  {x, \textrm{T} \bouchon{59} \{\nu\} \Cons \Sigma \rightarrow e}
\end{mathparpagebreakable}


\subsection{Intervalles bien form�s}
\label{intervalles_bien_formes}
%%-*-latex-*-

\begin{mathparpagebreakable}
% INTEGER
%
\inferrule
  {}
  {\bouchon{42} \textsf{MIN} \,\wild\, \textsf{\textbf{..}} \,\wild\,
    \textsf{MAX}}

\inferrule
  {}
  {\bouchon{42} \textsf{MIN} \,\wild\, \textsf{\textbf{..}} \,\wild\,
      (\textsf{Val} \, (\textsf{Int} \,\wild\!))}

\inferrule
  {}
  {\bouchon{42} (\textsf{Val} \, (\textsf{Int} \,\wild\!)) \,\wild\,
    \textsf{\textbf{..}} \,\wild\, \textsf{MAX}}

\inferrule
  {n \leqslant m}
  %----
  {\bouchon{42} (\textsf{Val} \, (\textsf{Int} \, n)) \leqslant
    \textsf{\textbf{..}} \leqslant (\textsf{Val} \, (\textsf{Int} \,
    m))}

\inferrule
  {n < m}
  %-----
  {\bouchon{42} (\textsf{Val} \, (\textsf{Int} \, n)) \leqslant
    \textsf{\textbf{..}} < (\textsf{Val} \, (\textsf{Int} \, m))}

\inferrule
  {n < m}
  %------
  {\bouchon{42} (\textsf{Val} \, (\textsf{Int} \, n)) <
    \textsf{\textbf{..}} \leqslant (\textsf{Val} \, (\textsf{Int} \,
    m))}

\inferrule
  {n+1 < m}
  %--------
  {\bouchon{42} (\textsf{Val} \, (\textsf{Int} \, n)) <
    \textsf{\textbf{..}} < (\textsf{Val} \, (\textsf{Int} \, m))}

% REAL
%
\inferrule
  {f_0 \lhd [(\textsf{Some} \, \, \textrm{"mantissa"},
              \textsf{Int}\,m_0); 
             (\textsf{Some} \, \, \textrm{"base"},
              \textsf{Int} \, b_0);\\\\
   (\textsf{Some} \, \, \textrm{"exponent"}, \textsf{Int} \, e_0)]\\
   f_1 \lhd [(\textsf{Some} \, \, \textrm{"mantissa"}, \textsf{Int} \,
     m_1); (\textsf{Some} \, \, \textrm{"base"}, \textsf{Int} \, b_1);\\\\
     (\textsf{Some} \, \, \textrm{"exponent"}, \textsf{Int} \, e_1)]\\\\
   m_0 \cdot b_0^{e_0} \leqslant m_1 \cdot b_1^{e_1}}
  %------
  {\bouchon{42} (\textsf{Val} \bob f_0 \bcb\!) \, \leqslant
    \textsf{\textbf{..}} \leqslant \, (\textsf{Val} \bob f_1 \bcb\!)}

\inferrule
  {f_0 \lhd [(\textsf{Some} \, \, \textrm{"mantissa"}, \textsf{Int} \,
      m_0); (\textsf{Some} \, \, \textrm{"base"}, \textsf{Int} \,
      b_0);\\\\
      (\textsf{Some} \, \, \textrm{"exponent"}, \textsf{Int} \,
      e_0)]\\
   f_1 \lhd [(\textsf{Some} \, \, \textrm{"mantissa"}, \textsf{Int} \,
     m_1); (\textsf{Some} \, \, \textrm{"base"}, \textsf{Int} \,
     b_1);\\\\
     (\textsf{Some} \, \, \textrm{"exponent"}, \textsf{Int} \, e_1)]\\\\\
   m_0 \cdot b_0^{e_0} < m_1 \cdot b_1^{e_1}}
  %------
  {\bouchon{42} (\textsf{Val} \bob f_0 \bcb\!) \,\wild\,
    \textsf{\textbf{..}} \,\wild\, (\textsf{Val} \bob f_1 \bcb\!)}

\inferrule
  {f_1 \lhd [(\textsf{Some} \, \, \textrm{"mantissa"}, \textsf{Int} \,
      m_1); (\textsf{Some} \, \, \textrm{"base"}, \textsf{Int} \,
      \wild\!);\\\\
      (\textsf{Some} \, \, \textrm{"exponent"}, \textsf{Int}
      \, \wild\!)]\\\\
   0 < m_1}
  %-----
  {\bouchon{42} (\textsf{Val 0.0}) \,\wild\, \textsf{\textbf{..}}
    \,\wild\, (\textsf{Val} \bob f_1 \bcb\!)}

\inferrule
  {f_0 \lhd [(\textsf{Some} \, \, \textrm{"mantissa"}, \textsf{Int} \,
      m_0); (\textsf{Some} \, \, \textrm{"base"}, \textsf{Int} \,
      \wild\!);\\\\
      (\textsf{Some} \, \, \textrm{"exponent"}, \textsf{Int}
      \, \wild\!)]\\\\
   m_0 < 0}
  %-------
  {\bouchon{42} (\textsf{Val} \bob f_0 \bcb\!) \,\wild\,
    \textsf{\textbf{..}} \,\wild\, (\textsf{Val 0.0})}

\inferrule
  {f_1 \lhd [(\textsf{Some} \, \, \textrm{"mantissa"}, \textsf{Int} \,
      \wild\!); (\textsf{Some} \, \, \textrm{"base"}, \textsf{Int} \,
      \wild\!);\\\\
      (\textsf{Some} \, \, \textrm{"exponent"}, \textsf{Int}
      \, \wild\!)]}
  %-----
  {\bouchon{42} (\textsf{Val MINUS-INFINITY}) \,\wild\,
    \textsf{\textbf{..}} \,\wild\, (\textsf{Val} \bob f_1 \bcb\!)}

\inferrule
  {f_0 \lhd [(\textsf{Some} \, \, \textrm{"mantissa"}, \textsf{Int} \,
      \wild\!); (\textsf{Some} \, \, \textrm{"base"}, \textsf{Int} \,
      \wild\!);\\\\
      (\textsf{Some} \, \, \textrm{"exponent"}, \textsf{Int}
      \, \wild\!)]}
  %------
  {\bouchon{42} (\textsf{Val} \bob f_0 \bcb\!) \,\wild\,
    \textsf{\textbf{..}} \,\wild\, (\textsf{Val PLUS-INFINITY})}

\inferrule
  {}
  {\bouchon{42} (\textsf{Val MINUS-INFINITY}) \,\wild\,
    \textsf{\textbf{..}} \,\wild\, (\textsf{Val PLUS-INFINITY})}

\inferrule
  {\textrm{String.length} \, s_0 = 1 \qquad \textrm{String.length} \, s_1 = 1\\
   \bouchon{42} \textsf{Val} \, (\textsf{Int} \, (\textrm{Char.code}
   \, (s_0.[0]))) \leqslant \textsf{\textbf{..}} \leqslant
   \textsf{Val} \, (\textsf{Int} \, (\textrm{Char.code} \, (s_1.[0])))}
  %-----
  {\bouchon{42} \textsf{Val} \, (\textsf{CharString} \, (s_0))
    \leqslant \textsf{\textbf{..}} \leqslant \textsf{Val} \,
    (\textsf{CharString} \, (s_1))}

% Identity of lower bound and upper bound
%
\inferrule
  {}
  {\bouchon{42} e \leqslant \textsf{\textbf{..}} \leqslant e}

\end{mathparpagebreakable}



\section{R�duction des contraintes}
%%-*-latex-*-

\subsection{R�duction des intersections de contraintes \textsf{SIZE}}
\label{reduction_des_intersections_de_contraintes_SIZE}
%%-*-latex-*-

\begin{mathparpagebreakable}
\inferrule
  {\Gamma(x) \lhd (\alpha, \tau, \SList{$(\textrm{T}_k,
      \{\nu_k\})$}{k}{q})\\
   \forall k \in [1..q].x, \textrm{T}_k \bouchon{91} \nu_k \rightarrow
   \overline\nu_k\\
   \bouchon{90} \domain{\mathcal{A}} \Gamma \rightarrow
   \domain{\overline{\mathcal{A}}} \overline\Gamma\\
   \overline\gamma \triangleq x \mapsto (\alpha, \tau,
   \SList{$(\textrm{T}_k, \{\overline\nu_k\})$}{k}{q})}
%-----
  {\bouchon{90} \domain{\{x\} \uplus \mathcal{A}} \Gamma \rightarrow
    \domain{\overline{\mathcal{A}}} \overline\Gamma \oplus
    \overline\gamma}

\inferrule
  {}
  {\bouchon{90} \domain{\varnothing} \Gamma \rightarrow
    \domain{\varnothing} \Gamma}

\inferrule
  {\textsf{None} \bouchon{74} \Sigma \rightarrow \Sigma_0\\
   \bouchon{72} \Sigma_0 \rightarrow \nu_0\\
   x, \textrm{T} \bouchon{54} \nu_0 \mid \emptyL \rightarrow \nu_1\\
   \bouchon{92} \nu_1 \rightarrow \overline\nu}
  %-----
  {x, \textrm{T} \bouchon{91} \textsf{Inter} \, (\Sigma) \rightarrow
    \overline\nu}

\inferrule
  {\nu \nlhd \textsf{Inter} \, \wild}
  %-----
  {x, \textrm{T} \bouchon{91} \nu \rightarrow \nu}

\end{mathparpagebreakable}


\subsection{Normalisation des intersections de contraintes \textsf{SIZE}}
\label{normalisation_des_intersections_de_contraintes_SIZE}
%%-*-latex-*-

\begin{mathparpagebreakable}
%%
\inferrule
  {\textsf{Some} \, (\nu) \bouchon{74} \Sigma \rightarrow
    \overline\Sigma}
  %----------------
  {\textsf{None} \bouchon{74}  \{\textsf{SIZE} \, \{\nu\}\} \Cons
    \Sigma \rightarrow \overline\Sigma}

\inferrule
  {\textsf{None} \bouchon{74} \textrm{merge} \, (\nu_0, \nu_1)
    \rightarrow \overline\Sigma_0\\
   \textsf{Some} \, (\textsf{Inter} \, \overline\Sigma_0) \bouchon{74}
   \Sigma \rightarrow \overline\Sigma_1}
  %-----
  {\textsf{Some} \, (\nu_1) \bouchon{74}  \{\textsf{SIZE} \,
    \{\nu_0\}\} \Cons \Sigma \rightarrow \overline\Sigma_1}

\inferrule
  {\nu \nlhd \textsf{SIZE} \, \wild\\
   s \bouchon{74} \Sigma \rightarrow \overline\Sigma}
  %-----
  {s \bouchon{74} \{\nu\} \Cons \Sigma \rightarrow \{\nu\} \Cons
    \overline\Sigma}

\inferrule
  {}
  {\textsf{Some} \, (\nu) \bouchon{74} \emptyL \rightarrow
    [\{\textsf{SIZE} \, \{\nu\}\}]}

\inferrule
  {}
  {\textsf{None} \bouchon{74} \emptyL \rightarrow \emptyL}

\end{mathparpagebreakable}


\subsection{�limination des contraintes \textsf{FROM} si \textsf{SIZE}(0)}
\label{elimination_des_contraintes_FROM_si_SIZE_0}
%%-*-latex-*-

\begin{mathparpagebreakable}
\inferrule
  {\bouchon{76} \Sigma \rightarrow \overline\Sigma\\
   \bouchon{72} \overline\Sigma \rightarrow \overline\nu}
  %-----
  {\bouchon{92} \textsf{Inter} \, (\Sigma) \rightarrow \overline\nu}

\inferrule
  {\nu \nlhd \textsf{Inter} \, \wild}
  %-----
  {\bouchon{92} \nu \rightarrow \nu}

% SIZE (0)  
%
\inferrule
  {\{\textsf{SIZE} \, \{\textsf{Value} \, (\textsf{Int} \, 0)\}\} \in
    \Sigma\\
   \bouchon{78} \Sigma \rightarrow \overline\Sigma}
  %-----
  {\bouchon{76} \Sigma \rightarrow \overline\Sigma}

% SIZE (_ _ .. _ _)
%
\inferrule
  {\{\textsf{SIZE} \, \{e_0 \, b_0 \textbf{..} b_1 \, e_1\}\} \in
    \Sigma\\
   \nu \triangleq \textsf{Val} \, (\textsf{Int} \, 0) \leqslant
   \textbf{..} \leqslant \textsf{Val} \, (\textsf{Int} \, 0)\\
   \bouchon{69} \nu \equiv e_0 b_0 \textbf{..} b_1 e_1\\
   \bouchon{78} \Sigma \rightarrow \overline\Sigma}
  %-----
  {\bouchon{76} \Sigma \rightarrow \overline\Sigma}

\inferrule
  {\forall \sigma \listin \Sigma.\sigma \nlhd \{\textsf{SIZE} \,
    \{\textsf{Value} \, (\textsf{Int} \, 0)\}\}\\
   \forall \sigma \listin \Sigma.(\sigma \lhd \{\textsf{SIZE} \, \{e_0
   \, b_0 \textbf{..} b_1 \, e_1\}\} \Rightarrow \neg(\bouchon{69} \nu
   \equiv e_0 b_0 \textbf{..} b_1 e_1))}
  %-----
  {\bouchon{76} \Sigma \rightarrow \Sigma}

\inferrule
  {\bouchon{78} \Sigma \rightarrow \overline\Sigma}
  %-----
  {\bouchon{78} \{\textsf{FROM} \, \sigma\} \Cons \Sigma \rightarrow
    \overline\Sigma}

\inferrule
  {\nu \nlhd \textsf{FROM} \, \wild\\
   \bouchon{78} \Sigma \rightarrow \overline\Sigma}
  %-----
  {\bouchon{78} \{\nu\} \Cons \Sigma \rightarrow \{\nu\} \Cons
    \overline\Sigma}

\inferrule
  {}
  {\bouchon{78} \emptyL \rightarrow \emptyL}

\end{mathparpagebreakable}


\subsection{R�duction des contraintes}

\begin{mathparpagebreakable}
  \inferrule
    {\Gamma(x) \lhd (\alpha, \tau, \SList{$(\textrm{T}_k,
        \{\nu_k\})$}{k}{q})\\
     \forall k \in [1..q].x, \textrm{ T}_k \bouchon{54} \nu_k \mid
     \emptyL \rightarrow \overline\nu_k\\
     \bouchon{94} \domain{\mathcal{ A}} \Gamma \rightarrow
     \domain{\overline{\mathcal{A}}} \overline\Gamma\\
     \overline\gamma \triangleq x \mapsto (\alpha, \tau,
     \SList{$(\textrm{T}_k, \{\overline\nu_k\})$}{k}{q})}
    %-----
    {\bouchon{94} \domain{\{x\} \uplus \mathcal{A}} \Gamma
      \rightarrow \domain{\overline{\mathcal{A}}} \overline\Gamma
      \oplus \overline\gamma}

  \inferrule
     {}
     {\bouchon{94} \domain{\varnothing} \Gamma \rightarrow
       \domain{\varnothing} \Gamma}

% Intersection

% Non empty constraint list
%
  \inferrule
    {x, \textrm{T} \bouchon{54} \nu \mid \emptyL \rightarrow
      \overline\nu_0\\
      x, \textrm{T} \bouchon{54} \textsf{Inter} \, (\Sigma) \mid
      [\{\overline\nu_0\}] \rightarrow \overline\nu_1}
    %-----
    {x, \textrm{T} \bouchon{54} \textsf{Inter} \, (\{\nu\} \Cons
      \Sigma) \mid \emptyL \rightarrow \overline\nu_1}

  \inferrule
    {\overline\Sigma_0 \nlhd \emptyL\\
     x, \textrm{T} \bouchon{54} \nu \mid \overline\Sigma_0 \rightarrow
     \textsf{Inter} \, (\overline\Sigma_1)\\
     x, \textrm{T} \bouchon{54} \textsf{Inter} \, (\Sigma) \mid
     \overline\Sigma_1 \rightarrow \overline\nu_1}
    %----
    {x, \textrm{T} \bouchon{54} \textsf{Inter} \, (\{\nu\} \Cons
      \Sigma) \mid \overline\Sigma_0 \rightarrow \overline\nu_1}

% Empty constraint list
%
  \inferrule
    {}
    {x, \textrm{T} \bouchon{54} \textsf{Inter} \, \emptyL \mid
    \overline\Sigma \rightarrow \textsf{Inter} \, (\overline\Sigma)}

% Value constraint for REAL
%
  \inferrule
    {\textrm{"REAL"}, \textrm{T} \bouchon{73} \textsf{Value} \,
      (\overline{v}) \mid \overline\Sigma \rightarrow \overline\nu\\
     \overline{v} \triangleq \textrm{Real.normalize} \, (v)}
    %-----
    {\textrm{"REAL"}, \textrm{T} \bouchon{54} \textsf{Value} \, (v)
      \mid \overline\Sigma \rightarrow \overline\nu}

% Value constraint for other types
%
  \inferrule
    {x \neq \REAL\\
      x, \textrm{T} \bouchon{73} \textsf{Value} \, (v) \mid
      \overline\Sigma \rightarrow \overline\nu}
    %-----
    {x, \textrm{T} \bouchon{54} \textsf{Value} \, (v) \mid
      \overline\Sigma \rightarrow \overline\nu}

% Value Range Constraint 

% .....for INTEGER and CharString
%
  \inferrule
    {\textrm{T} \lhd \wildINTEGER \mid \textsf{CharString} \, \wild\\
     x, \textrm{T} \bouchon{44} \nu \mid \overline\Sigma \rightarrow
     \overline\nu}
    %-----
    {x, \textrm{T} \bouchon{54} \wild \wild \textsf{\textbf{..}} \wild
      \wild \AS \nu \mid \overline\Sigma \rightarrow \overline\nu}

% .....for REAL
%
  \inferrule
    {x = \textrm{"REAL"}\\
     \bouchon{50} \nu \rightarrow \nu_0\\
     x, \textrm{T} \bouchon{44} \nu_0 \mid \overline\Sigma \rightarrow
     \overline\nu}
    %-----
    {x, \textrm{T} \bouchon{54} \wild \wild \textsf{\textbf{..}} \wild
      \wild \AS \nu \mid \overline\Sigma \rightarrow \overline\nu}

% Size Range Constraint
%
  \inferrule
    {\textrm{T} \lhd \wildBITSTRING \mid \textsf{OCTET STRING} \\
     \mid (\textsf{SET OF} \mid \textsf{SEQUENCE OF}) \wild \wild
     \wild \, \mid \textsf{CharString} \, \wild\\
     \sigma_0 \triangleq \{\textsf{Inter} \, [\textsf{Val} \,
       (\textsf{Int} \, 0) \, \leqslant \textsf{\textbf{..}} \leqslant
       \, \textsf{MAX}; \sigma']\}\\
     \INTEGER, \emptyINTEGER \bouchon{55} \sigma_0 \mid \emptyL
     \rightarrow \overline\sigma_0\\
     x, \textrm{T} \bouchon{73} \textsf{SIZE} \, (\overline\sigma_0)
     \mid \overline\Sigma \rightarrow \overline\nu}
    %-----
    {x, \textrm{T} \bouchon{54} \textsf{SIZE} \, (\sigma') \mid
      \overline\Sigma \rightarrow \overline\nu}

% Alphabet Limitation Constraint
%
  \inferrule
    {\textrm{T} \lhd \textsf{CharString} \, \wild\\
      x, \textrm{T} \bouchon{55} \sigma' \mid \emptyL \rightarrow
      \overline\sigma'\\
      x, \textrm{T} \bouchon{73} \textsf{FROM} \, (\overline\sigma')
      \mid \overline\Sigma \rightarrow \overline\nu}
    %-----
    {x, \textrm{T} \bouchon{54} \textsf{FROM} \, (\sigma') \mid
      \overline\Sigma \rightarrow \overline\nu}

  \inferrule
    {x, \textrm{T} \bouchon{54} \nu \mid \emptyL \rightarrow
      \overline\nu\\
     x, \textrm{T} \bouchon{55} \sigma \rightarrow \overline\sigma}
    %-----
    {x, \textrm{T} \bouchon{55} \{\nu\} \uplus \sigma \rightarrow
      \{\overline\nu\} \cup \overline\sigma}

  \inferrule
    {}
    {x, \textrm{T} \bouchon{55} \{\} \rightarrow \{\}}

  \inferrule
  % Normalisation MIN et MAX
    {x, \textrm{T} \bouchon{45} \nu \mid \overline\Sigma \rightarrow
      \nu_0\\
  % Bonne formation de l'intervalle
     \bouchon{42} \nu_0\\
  % Intersection de l'intervalle et du reste
     x, \textrm{T} \bouchon{73} \nu_0 \mid \overline\Sigma \rightarrow
     \overline\nu}
    %-----
    {x, \textrm{T} \bouchon{44} \nu \mid \overline\Sigma \rightarrow
      \overline\nu}
\end{mathparpagebreakable}

\subsection{Intersection de contraintes}
\label{intersection_de_contraintes}
%%-*-latex-*-

\begin{mathparpagebreakable}
%%
\inferrule
  {x, \textrm{T} \bouchon{70} \nu \mid \Sigma \rightarrow
    \overline\Sigma\\
   \bouchon{72} \overline\Sigma \rightarrow \overline\nu}
  %-----
  {x, \textrm{T} \bouchon{73} \nu \mid \Sigma \rightarrow
    \overline\nu}

% Value constraint for INTEGER and CharString
%
\inferrule
  {\textrm{T} \lhd \wildINTEGER \mid \textsf{CharString} \, \wild\\\\
    \OR x = \REAL\\
   \nu_0 \lhd \wild \wild \textbf{..} \wild \wild \mid \textsf{Value}
   \, \wild\\
   \nu_1 \lhd \wild \wild \textbf{..} \wild \wild \mid \textsf{Value}
   \, \wild\\
   \bouchon{71} \nu_0 \cap \nu_1 \rightarrow \nu_2}
  %-----
  {x, \textrm{T} \bouchon{70} \nu_0 \mid [\{\nu_1\}] \sqcup
    \overline\Sigma \rightarrow [\{\nu_2\}] \listunion \overline\Sigma}

% Value constraint for other types
%
\inferrule
  {\{\textsf{Value} \, (v)\} \listin \overline\Sigma}
  %-----
  {x, \textrm{T} \bouchon{70} \textsf{Value} \, (v) \mid
    \overline\Sigma \rightarrow \overline\Sigma}

% SIZE
%
\inferrule
  {\bouchon{72} \textrm{merge} \, (\nu_0, \nu_1) \rightarrow \nu_2\\
   x, \textrm{T} \bouchon{54} \textsf{SIZE} \, \{\nu_2\} \mid \emptyL
   \rightarrow \overline\nu}
  %-----
  {x, \textrm{T} \bouchon{70} \textsf{SIZE} \, \{\nu_0\} \mid
    [\{\textsf{SIZE} \, \{\nu_1\}\}] \sqcup \overline\Sigma
    \rightarrow [\{\overline\nu\}] \listunion \overline\Sigma}

% FROM
%
\inferrule
  {\bouchon{72} \textrm{merge} \, (\nu_0, \nu_1) \rightarrow \nu_2\\
   x, \textrm{T} \bouchon{54} \textsf{FROM} \, \{\nu_2\} \mid \emptyL
   \rightarrow \overline\nu}
  %------
  {x, \textrm{T} \bouchon{70} \textsf{FROM} \, \{\nu_0\} \mid
    [\{\textsf{FROM} \, \{\nu_1\}\}] \sqcup \overline\Sigma
    \rightarrow [\{\overline\nu\}] \listunion \overline\Sigma}

% Others
%
\inferrule
  {(\nu_0, \nu_1) \nlhd ((\wild \wild \textbf{..} \wild \wild \mid
    \textsf{Value} \, \wild\!), (\wild \wild \textbf{..} \wild \wild
    \mid \textsf{Value} \, \wild\!)) \\\\
    \mid (\textsf{SIZE} \, \wild\!, \textsf{SIZE} \, \wild\!) \mid
    (\textsf{FROM} \, \wild\!, \textsf{FROM} \, \wild\!)\\\\
   x, \textrm{T} \bouchon{70} \nu_0 \mid \Sigma \rightarrow \overline\Sigma_0}
  %----
  {x, \textrm{T} \bouchon{70} \nu_0 \mid [\{\nu_1\}] \sqcup
    \overline\Sigma \rightarrow [\{\nu_1\}] \listunion
    \overline\Sigma_0}

\inferrule
  {}
  {x, \textrm{T} \bouchon{70} \nu_0 \mid \emptyL \rightarrow
    [\{\nu_0\}]}

\inferrule
  {}
  {\bouchon{72} [\{\nu\}] \rightarrow \nu}

\inferrule
  {\Sigma \nlhd [\wild]}
  %-----
  {\bouchon{72} \Sigma \rightarrow \textsf{Inter} \, (\Sigma)}

\end{mathparpagebreakable}



\section{Sous-types bien form�s}

\begin{mathpar}
  \inferrule
% R�duction des INCLUDES
  {\bouchon{89} \domain{\mathcal{A}} \Gamma \rightarrow
    \domain{\mathcal{A}_0} \Gamma_0\\
% Forme normale conjontive
   \bouchon{62} \domain{\mathcal{A}_0} \Gamma_0 \rightarrow
   \domain{\mathcal{A}_1} \Gamma_1\\
% D�pliage des valeurs dans les contraintes
   \domain{\mathcal{B}} \Delta \bouchon{43} \domain{\mathcal{A}_1}
   \Gamma_1 \rightarrow \domain{\mathcal{A}_2} \Gamma_2\\
% R�duction compl�te des contraintes internes
   \bouchon{20} \domain{\mathcal{A}_2} \Gamma_2 \rightarrow
   \domain{\mathcal{A}_3} \Gamma_3\\
% Types bien form�s? 
% Eg T ::= CHOICE {a REAL, b NULL} 
%                 (WITH COMPONENTS {b ABSENT})
   \Vdash \domain{\mathcal{A}_3} \Gamma_3\\
% R�duction des intersections de contraintes SIZE
   \bouchon{90} \domain{\mathcal{A}_3} \Gamma_3 \rightarrow
   \domain{\mathcal{A}_4} \Gamma_4\\
% R�duction des intersections de contraintes
   \bouchon{94} \domain{\mathcal{A}_4} \Gamma_4 \rightarrow \wild}
  %-----
  {\domain{\mathcal{B}} \Delta \bouchon{93} \domain{\mathcal{A}}
    \Gamma}
\end{mathpar}
